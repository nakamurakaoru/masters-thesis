\documentclass[unicode,mathserif]{beamer}
\usetheme{Madrid}
\usefonttheme{professionalfonts}
%%%%%%%%%%%%%%%%%%%%%%%%%%%%%%%%%%%%%%%%%%%%%%%%%%%%%%%%%%%%%%%%%%%%%%%%%%%
\usepackage[orientation=landscape, size=a0, scale=1.4]{beamerposter}
\usepackage{luatexja}
%\usepackage{amsmath}
%\usepackage{amsthm}
%\usepackage{ascmac}
%\newtheoremstyle{mystyle}% % Name
%    {}%                      % Space above
%    {}%                      % Space below
%    {\normalfont}%           % Body font
%    {}%                      % Indent amount
%    {\bfseries}%             % Theorem head font
%    {}%                      % Punctuation after theorem head
%    { }%                     % Space after theorem head, ' ', or \newline
%    {}%                      % Theorem head spec (can be left empty, meaning `normal')
%\theoremstyle{mystyle}
%\usepackage{amssymb}
%\usepackage{ascmac}
%\usepackage{txfonts}
%% packages
\usepackage{ascmac}

\usepackage{amsmath}
\usepackage{amssymb}

\usepackage{cite}
\usepackage{graphicx}
\usepackage{enumerate}
\usepackage{url}
\usepackage{mathrsfs}
\usepackage{eucal}
\usepackage{listings}

% %% amsthm
% \usepackage{amsthm}
% \theoremstyle{plain}
% \newtheorem{theorem}{定理}[section]
% \newtheorem{lemma}[theorem]{補題}
% \newtheorem{condition}[theorem]{条件}
% \newtheorem{axiom}[theorem]{公理}
% \theoremstyle{definition}
% \newtheorem{definition}[theorem]{定義}
% \newtheorem{exercise}[theorem]{演習}
% \newtheorem{remark}[theorem]{注意}
% \newtheorem{example}[theorem]{例}
% \newtheorem*{definition*}{定義}
% \newtheorem*{exercise*}{演習}
% \newtheorem*{remark*}{注意}
% \newtheorem*{example*}{例}

%% xcolor
\usepackage{xcolor}
\definecolor{darkgreen}{rgb}{0,0.5,0}
\definecolor{darkred}{rgb}{0.5,0,0}
\definecolor{darkyellow}{rgb}{0.3,0.3,0}
\definecolor{darkorange}{rgb}{0.4,0.2,0}
\definecolor{myviolet}{rgb}{0.6,0.0,0.65}
\definecolor{myblue}{rgb}{0.1,0.0,0.8}
\definecolor{mygreen}{rgb}{0.2,0.7,0.1}
\definecolor{mycomment}{rgb}{0.7,0.2,0.2}
\definecolor{myred}{rgb}{0.8,0.0,0.0}
\definecolor{myorange}{rgb}{0.6,0.2,0.2}

% \pgfdeclarelayer{background}
% \pgfdeclarelayer{foreground}
% \pgfsetlayers{background,main,foreground}

\lstdefinelanguage{coq}[]{Caml}{
keywords=[1]{Section,Definition,Defined,CoInductive,Coercion,Inductive,Fixpoint,
  Parameter,Module,Import,Record,Structure,Axiom,Lemma,Theorem,Notation,
  Reserved,End,Proof,Goal,Qed,Variable,Variables,Hypothesis,Let,Program,Canonical},
keywordstyle=\color{myviolet}\ttfamily,
morekeywords=[2]{match,with,end,Set,Prop,Type,fun,of,let,in,struct,if,is,then,else,return},
keywordstyle=[2]\color{mygreen}\ttfamily,
morekeywords=[3]{move},
keywordstyle=[3]\color{myblue}\ttfamily,
morekeywords=[3]{field,lra},
keywordstyle=[3]\color{myred}\ttfamily
}

\lstset{
language=coq,
columns=fullflexible,
keepspaces,
basicstyle=\small\ttfamily,
identifierstyle=\color{black}\ttfamily,
commentstyle=\color{mycomment}\it\ttfamily,
morecomment=[n]{(*}{*)},
morestring=[b][\color{myorange}\ttfamily]",
showstringspaces=false,
extendedchars=true,
literate=
{forall}{$\forall$}1
{!=}{$\neq$}1
{<>}{$\neq$}1
{<=}{$\leq$}1
{<=m}{$\leq_m$}1
{<=l}{$\leq_l$}1
{->}{$\to$}1
{<->}{$\leftrightarrow$}1
{==>}{$\Longrightarrow$}1
{~~}{$\neg$}1
{=>}{$\Rightarrow$}1
{\#|}{\#|}1
{exists}{$\exists$}1
{\\in}{$\in$}1
{\\sigma_}{$\sigma$}1
{\\omega_}{$\omega$}1
{\\delta}{$\delta$}1
{/\\}{$\land$}1
}

\let\L=\lstinline

%% definitions
%% semantic markup commands
\newcommand\newterm[1]{\emph{#1}}

%% copiable underscore
\newcommand{\us}{\char`\_}

%% basic math notations
\newcommand\imp{\Rightarrow}
\newcommand\tuple[1]{\langle{#1}\rangle}
\newcommand\enum[1]{\{ #1 \}}
\newcommand\comprh[2]{\left\{{#1}\ \middle|\ {#2}\right\}}
%\newcommand\comprh[2]{\left\{{#1}\ :\ {#2}\right\}}
\newcommand\defarrow{\overset{def}\Longleftrightarrow}
\newcommand\defeq{:=}

%% matrix notations
\newcommand\mx[2]{\left[ \begin{array}{#1} #2 \end{array} \right]}
\newcommand\mxc{\mx{cccccccccc}}
\newcommand\mxr{\mx{r}}

%% calculus notations
\newcommand\seq[2][n]{\left({#2}\right)_{{#1}\ge0}}
\newcommand\seqa[2][n]{\left({#2}_{#1}\right)_{{#1}\ge0}}
\newcommand\limseq[1][n]{\lim_{{#1}\to\infty}}
\newcommand\ser[2][k]{\sum_{{#1}=0}^\infty{#2}}

%% sets of numbers
\newcommand\N{\mathbb N}
\newcommand\Z{\mathbb Z}
\newcommand\Q{\mathbb Q}
\newcommand\R{\mathbb R}
\newcommand\Rpos{\mathbb R_{>0}}
\newcommand\Rneg{\mathbb R_{<0}}
\newcommand\Rnneg{\mathbb R_{\geq0}}
\newcommand\Rnpos{\mathbb R_{\leq0}}
\newcommand\Rext{\mathbb R \cup \enum{\pm\infty}}

%% set theory
\DeclareMathOperator{\dom}{dom}
\DeclareMathOperator{\ran}{ran}
\DeclareMathOperator{\id}{id}
\newcommand\power{\mathscr{P}}

%%%%%%%%%%%%%%%%%%%%%%%%%%%%%%%%%%%%%%%%%%%%%%%%%%%%%%%%%%%%%%%%%%%%%%%%%%%
%\setbeamertemplate{theorems}[numbered]  %% 定理に番号をつける
%\newtheorem{df}{$\textrm{Definition}$}[section]
%\newtheorem{ex}[df]{$\textrm{Example}$}
%\newtheorem{prop}[df]{$\textrm{Proposition}$}
%\newtheorem{lem}[df]{$\textrm{Lemma}$}
%\newtheorem{cor}[df]{$\textrm{Corollary}$}
%\newtheorem{rmk}[df]{$\textrm{Remark}$}
%\newtheorem{thm}[df]{$\textrm{Theorem}$}
%\newtheorem{axi}[df]{$\textrm{Axiom}$}
%% always %%%%%%%%%%%%%%%%%%%%%%%%%%%%%%%%%%%%%%%%%%%%%%%%%%%%%%%%%%%%%%%%%%
\newcommand{\Lra}{\Longrightarrow}
\newcommand{\Llra}{\Longleftrightarrow}
\newcommand{\lra}{\leftrightarrow}
\newcommand{\ra}{\rightarrow}
\newcommand{\ol}[1]{{\overline{#1}}}
\newcommand{\ul}[1]{{\underline{#1}}}
%\newcommand\infotheo{\textsc{InfoTheo}}
%\newcommand\coq{\textsc{Coq}}
%\newcommand\mathcomp{\textsc{MathComp}}
%
%\newcommand\coqin[1]{\mintinline{ssr}{#1}}
%
%\newcommand\Set{\mathbf{Set}}
%\newcommand\Cat{\mathbf{Cat}}
%\newcommand\Hilb{\mathbf{FdHilb}}
%\newcommand\CNOT{\mathtt{CNOT}}
%% sets %%%%%%%%%%%%%%%%%%%%%%%%%%%%%%%%%%%%%%%%%%%%%%%%%%%%%%%%%%%%%%%%%%%%
\newcommand{\K}{\mathbb{K}}
\newcommand{\D}{\mathbb{D}}
%% q-analogue %%%%%%%%%%%%%%%%%%%%%%%%%%%%%%%%%%%%%%%%%%%%%%%%%%%%%%%%%%%%%%%
\newcommand{\qcoe}[2]{\left[\begin{array}{ccc}#1\\#2\end{array}\right]}
%%%%%%%%%%%%%%%%%%%%%%%%%%%%%%%%%%%%%%%%%%%%%%%%%%%%%%%%%%%%%%%%%%%%%%%%%%%
\title{$q$-類似の Coq による形式化}
\author{中村薫}

\begin{document}
\begin{frame}[fragile]
    \begin{center}
      {\Huge $q$-類似の Coq による形式化} \\
      {\Large 中村薫} \qquad {\large 名古屋大学}
    \end{center}

	\begin{columns}[T]
		\begin{column}{0.33\columnwidth}
			\begin{block}{はじめに}
				\begin{itemize}
				\item $q$-類似の初等的な結果を Coq で形式化する 
				
					$\to$具体的にはTaylor展開の$q$-類似の形式化が目標 
				\item $q$-類似の定義, 定理, 証明は[Kac]による
				\item 形式化には mathcomp を用いている 
				\end{itemize}
			\end{block}
			
		\begin{block}{$q$-類似とは}
			以下の2つの条件をみたす数学の諸概念の一般化
			\begin{itemize}
				\item $q \to 1$とすると通常の数学に一致する
	          \item 実数パラメータ$q$, 実数上の関数$f$に対して
					\[
						D_q f(x) := \frac{f(qx) - f(x)}{(q - 1) x}
					\]
					で定義される$q$-微分に対してうまく振る舞う
			\end{itemize}
		\end{block}

		\begin{block}{$q$-微分}
			以下, $q$を$1$でない実数とする.
			\begin{itembox}{Definition [Kac] p1 (1.1), p2 (1.5)}
				関数$f : \R \to \R$に対して, $f(x)$の$q$-差分$d_q f(x)$と$q$-微分$D_q f(x)$を
				以下のように定める.  
				\[
					d_q f(x) := f (qx) - f(x), \quad
					D_q f(x) := \frac{d_q f(x)}{d_q x} = \frac{f(qx) - f(x)}{(q - 1) x}.
				\]
			\end{itembox}
		\end{block}
   
		\begin{block}{自然数の$q$-類似}
			$f(x) = x^n$を定義に沿って$q$-微分すると以下の通りである. 
			\[
				D_q x^n = \frac{(qx)^n - x^n}{(q - 1) x}
								= \frac{q^n - 1}{q - 1} \cdot \frac{x^n}{x}
								= \frac{q^n - 1}{q - 1} x^{n - 1}
			\]
			通常の微分では, $(x^n)' = n x^{n - 1}$となることと比較して, 
			自然数の$q$-類似を定める.
			\begin{itembox}{Definition [Kac] p2 (1.9)}.
				$n \in \N$について, $n$の$q$-類似$[n]$を
				\[
					[n] = \frac{q^n - 1}{q - 1} (= 1 + q + q^2 + \cdots q^{n - 1})
				\]
				と定義する. 
			\end{itembox}
		\end{block}

		\begin{block}{$(x - a)^n$の$q$-類似}
			$D_q(x-a)^n_q = [n](x-a)^{n-1}_q$をみたすように$(x - a)^n_q$の$q$-類似を定める. 
			\begin{itembox}{Definition [Kac] p8 Definition (3.4)}
			$x$, $a \in \R$, $n \in \N$に対して, $(x - a)^n$の$q$-類似$(x - a)^n_q$を, 
			\[
				(x - a)^n_q := \begin{cases}
					1 & \text{if}\ n = 0 \\
					(x - a) (x - qa) \cdots (x - q^{n - 1} a) & \text{if}\ n \ge 1
				\end{cases}
			\]
			と定義する.
			\end{itembox}
		\end{block}
		
		\begin{block}{$q$-Taylor展開}
			\begin{itembox}{Theorem [Kac] p12 Theorem 4.1}
				$f(x)$を, $N$次の実数係数多項式とする. 任意の$c\in\R$に対し, 
				\[
					f(x) = \sum_{j=0}^N (D_q^jf)(c)\frac{(x-c)^j_q}{[j]!}
						\left( [n]! := \begin{cases}
													1 & (n=0)\\
													[n]\times[n-1]\times\cdots\times[1] & (n\ge1)
												\end{cases}
						\right)
				\]
				が成り立つ.
			\end{itembox}
		\end{block}
	\end{column}
%%%%%%%%%%%%%%%%%%%%%%%%%%%%%%%%%%%%%%%%%%%%%%%%%%%%%%%%%%%%%%%%%%%%%%%%%%%
	\begin{column}{0.33\columnwidth}
		\begin{block}{mathcomp の構造}
			実数として mathcomp の ssrnum で定義されている{\tt rcfType}を用いている
			\begin{lstlisting}{Coq}
Variables (R : rcfType) (q : R). \end{lstlisting}			
			mathcomp の型には階層構造があり, より一般の型の構造を引き継ぐ
			\begin{align*}
  			{\tt eqType} &\to {\tt choiceType} \\
                   &\to {\tt zmodType} \to {\tt ringType} \to 
                          {\tt comRingType} \to {\tt comUnitRingType} \\
						&\to {\tt idomainType} \to {\tt fieldType}\\
                   &\to {\tt numFieldType} \to {\tt realFieldType} \to {\tt rcfType}
			\end{align*}
			{\tt rcfType}は特に{\tt ringType}を引き継いでいることが重要
			
			$\to$用いる補題のほとんどが{\tt ringType}に対するもの
		\end{block}
		
		\begin{block}{$q$-微分の形式化}
			\begin{lstlisting}{Coq}
Hypothesis Hq : q - 1 != 0.
Notation "f // g" := (fun x => f x / g x) (at level 40).
Definition dq (f : R -> R) x := f (q * x) - f x.
Definition Dq f := dq f // dq id. \end{lstlisting}
		\end{block}
		
		\begin{block}{$[n]$の形式化}
			\begin{lstlisting}{Coq}
Definition qnat n : R := (q ^ n - 1) / (q - 1).
Lemma Dq_pow n x : x != 0 ->
  Dq (fun x => x ^ n) x = qnat n * x ^ (n - 1).
Proof.
  move=> Hx.
  rewrite /Dq /dq /qnat.
  rewrite -{4}(mul1r x) -mulrBl expfzMl -add_div; last by apply mulf_neq0.
  rewrite [in x ^ n](_ : n = (n -1) +1) //; last by rewrite subrK.
  rewrite expfzDr ?expr1z ?mulrA -?mulNr ?red_frac_r ?add_div //.
  rewrite -{2}[x ^ (n - 1)]mul1r -mulrBl mulrC mulrA.
  by rewrite [in (q - 1)^-1 * (q ^ n - 1)] mulrC.
Qed. \end{lstlisting}
		\end{block}
		
		\begin{block}{$(x - a)^n_q$の形式化}
			\begin{lstlisting}{Coq}
Fixpoint qbinom_pos a n x :=
  match n with
  | 0 => 1
  | n0.+1 => (qbinom_pos a n0 x) * (x - q ^ n0 * a)
  end. 
Theorem Dq_qbinom_pos a n x : x != 0 ->
  Dq (qbinom_pos a n.+1) x = qnat n.+1 * qbinom_pos a n x. \end{lstlisting}
		\end{block}

		\begin{block}{関数から多項式へ}
			$x / x = 1$を計算するとき
			\begin{itemize}
				\item 実数 $\cdots$ $x \ne 0$が必要
				\item 多項式 $\cdots$ $x$は単項式なので自動的に $x \ne 0$(ゼロ多項式)
			\end{itemize}
			$\to$多項式で考えれば約分した後でも$x = 0$での値が求められる.
		\end{block}
		
		\begin{block}{mathcomp での多項式の構造}
			環上の多項式全体は環を成し, 加群の構造を持つ.
			
			このことは mathcomp の poly.v で形式化されている.
			\begin{lstlisting}{Coq}
poly : forall R : ringType, nat -> (nat -> R) -> {poly R} \end{lstlisting}			
			$\to${\tt ringType}の補題がそのまま使え, スカラー倍も扱える. 
			
			整域上の多項式では割り算ができる(mathcomp の polydiv.v).
			\begin{lstlisting}{Coq}
divp : forall R : idomainType, {poly R} -> {poly R} -> {poly R} \end{lstlisting}
			$\to${\tt divp}は多項式の割り算なので, 余りが$0$でない可能性がある. 
		\end{block}
	\end{column}
%%%%%%%%%%%%%%%%%%%%%%%%%%%%%%%%%%%%%%%%%%%%%%%%%%%%%%%%%%%%%%%%%%%%%%%%%%%
	\begin{column}{0.33\columnwidth}
		\begin{block}{多項式での再定義}
			\begin{itemize}
			\item $q$-差分
			\begin{lstlisting}{Coq}
Definition scale_var (p : {poly R}):=
 \poly_(i < size p) (q ^ i * p`_i).
Definition dqp p := scale_var p - p. \end{lstlisting}
		{\tt scale\_var} $\cdots$ {\tt (scale\_var p).[x] = p.[qx]}
			\item $q$-微分
			\begin{lstlisting}{Coq}
Definition Dqp p := dqp p %/ dqp 'X.
\end{lstlisting}
		{\tt Dqp'} $\cdots$ {\tt Dqp}を{\tt 'X}で約分した形
			\item $(x - a)^n_q$
			\begin{lstlisting}{Coq}
Fixpoint qbinom_pos_poly a n :=
  match n with
  | 0 => 1
  | n0.+1 => (qbinom_pos_poly a n0) * ('X - (q ^ n0 * a)%:P)
  end. \end{lstlisting}
  			\end{itemize}
		\end{block}
		
		\begin{block}{{\tt Dqp}が正しい割り算である保証}
%			{\tt Dqp}の定義からは余りが$0$でない可能性がある.
%			
%			$\to$実際に計算すると{\tt dqp p}は定数項が打ち消しあい, 
%			{\tt dqp 'X}は{\tt (q - 1) * 'X}となるので割り切れるはず. 
%			\begin{lstlisting}{Coq}
%Lemma Dqp_ok p : dqp 'X %| dqp p.
%\end{lstlisting}
			整域上の多項式全体は整域となるため, 商体を構成できる
			
			商体は mathcomp の fraction.v で形式化されている. 
			\begin{lstlisting}{Coq}
Local Notation tofrac := (@tofrac [idomainType of {poly R}]).
Local Notation "x %:F" := (tofrac x).

Theorem Dqp_ok_frac p : (dqp p)%:F / (dqp 'X)%:F = (Dqp p)%:F.
\end{lstlisting}
		\end{block}

		\begin{block}{各点での定義との関係}
			多項式として定義し直した{\tt Dqp'}, {\tt qbinom\_pos\_poly}も期待通り振る舞う.
%				\begin{lstlisting}{Coq}
%Definition ap_op_poly (D : (R -> R) -> (R -> R)) (p : {poly R}) :=
%  D (fun (x : R) => p.[x]).
%Notation "D # p" := (ap_op_poly D p) (at level 49).
%Lemma dqp_dqE p x : (dqp p).[x] = (dq # p) x. \end{lstlisting}
%				$\to${\tt p}に{\tt dqp}を適用した多項式の{\tt x}での値と
%				{\tt p \mapsto p.[x]}という関数に{\tt dq}を適用した関数の{\tt x}での値は等しい
%				\begin{lstlisting}{Coq}
%Lemma Dqp_Dqp'E p : Dqp p = Dqp' p.
%Lemma Dqp'_DqE p x : x != 0 -> (Dqp' p).[x] = (Dq # p) x. 
				\begin{lstlisting}{Coq}
Lemma qbinom_posE a n x :
  qbinom_pos a n x = (qbinom_pos_poly a n).[x].
Lemma Dqp'_qbinom_poly a n :
  Dqp' (qbinom_pos_poly a n.+1) = (qnat n.+1) *: (qbinom_pos_poly a n).
\end{lstlisting}
		\end{block}
		
		\begin{block}{$q$-Taylor展開の形式化}
			\begin{lstlisting}{Coq}
Fixpoint qfact n :=
  match n with
  | 0 => 1
  | n0.+1 => qfact n0 * qnat n0.+1
  end.
Theorem q_Taylorp n (f : {poly R}) c :
  (forall n, qfact n != 0) ->
  size f = n.+1 ->
  f = \sum_(0 <= i < n.+1)
    ((Dqp' \^ i) f).[c] *: (qbinom_pos_poly c i / (qfact i)%:P).
\end{lstlisting}
		\end{block}
		
		\begin{block}{今後の展望}
			現在開発中のライブラリ mathcomp analysis の利用
			\begin{itemize}
				\item $q \to 1$で通常の数学に戻ることの形式化
				\item 無限和に関する形式化
				
				$\to$$q$-Taylor展開の無限次への拡張, $q$-指数関数, $q$-三角関数
			\end{itemize}
		\end{block}
		
		\begin{block}{参考文献}
			[Kac] Victor Kac, Pokman Cheung, {\it{Quantum Calculus}}, Springer, 2001.
		\end{block}

		形式化したコードは github で公開している. 
		
		\url{https://github.com/nakamurakaoru/q-analogue/tree/thesis}
		
		また, 名古屋大学での修士論文も本ポスターの内容についてである. 
		
		\url{https://github.com/nakamurakaoru/masters-thesis/tree/main}
		
	\end{column}
	\end{columns}
\end{frame}
\end{document}