\documentclass[unicode,mathserif]{beamer}
\usetheme{Madrid}
%\usetheme{ZH}

\usepackage[orientation=landscape, size=a0, scale=1.4]{beamerposter}
\usepackage{luatexja}
%% packages
\usepackage{ascmac}

\usepackage{amsmath}
\usepackage{amssymb}

\usepackage{cite}
\usepackage{graphicx}
\usepackage{enumerate}
\usepackage{url}
\usepackage{mathrsfs}
\usepackage{eucal}
\usepackage{listings}

% %% amsthm
% \usepackage{amsthm}
% \theoremstyle{plain}
% \newtheorem{theorem}{定理}[section]
% \newtheorem{lemma}[theorem]{補題}
% \newtheorem{condition}[theorem]{条件}
% \newtheorem{axiom}[theorem]{公理}
% \theoremstyle{definition}
% \newtheorem{definition}[theorem]{定義}
% \newtheorem{exercise}[theorem]{演習}
% \newtheorem{remark}[theorem]{注意}
% \newtheorem{example}[theorem]{例}
% \newtheorem*{definition*}{定義}
% \newtheorem*{exercise*}{演習}
% \newtheorem*{remark*}{注意}
% \newtheorem*{example*}{例}

%% xcolor
\usepackage{xcolor}
\definecolor{darkgreen}{rgb}{0,0.5,0}
\definecolor{darkred}{rgb}{0.5,0,0}
\definecolor{darkyellow}{rgb}{0.3,0.3,0}
\definecolor{darkorange}{rgb}{0.4,0.2,0}
\definecolor{myviolet}{rgb}{0.6,0.0,0.65}
\definecolor{myblue}{rgb}{0.1,0.0,0.8}
\definecolor{mygreen}{rgb}{0.2,0.7,0.1}
\definecolor{mycomment}{rgb}{0.7,0.2,0.2}
\definecolor{myred}{rgb}{0.8,0.0,0.0}
\definecolor{myorange}{rgb}{0.6,0.2,0.2}

% \pgfdeclarelayer{background}
% \pgfdeclarelayer{foreground}
% \pgfsetlayers{background,main,foreground}

\lstdefinelanguage{coq}[]{Caml}{
keywords=[1]{Section,Definition,Defined,CoInductive,Coercion,Inductive,Fixpoint,
  Parameter,Module,Import,Record,Structure,Axiom,Lemma,Theorem,Notation,
  Reserved,End,Proof,Goal,Qed,Variable,Variables,Hypothesis,Let,Program,Canonical},
keywordstyle=\color{myviolet}\ttfamily,
morekeywords=[2]{match,with,end,Set,Prop,Type,fun,of,let,in,struct,if,is,then,else,return},
keywordstyle=[2]\color{mygreen}\ttfamily,
morekeywords=[3]{move},
keywordstyle=[3]\color{myblue}\ttfamily,
morekeywords=[3]{field,lra},
keywordstyle=[3]\color{myred}\ttfamily
}

\lstset{
language=coq,
columns=fullflexible,
keepspaces,
basicstyle=\small\ttfamily,
identifierstyle=\color{black}\ttfamily,
commentstyle=\color{mycomment}\it\ttfamily,
morecomment=[n]{(*}{*)},
morestring=[b][\color{myorange}\ttfamily]",
showstringspaces=false,
extendedchars=true,
literate=
{forall}{$\forall$}1
{!=}{$\neq$}1
{<>}{$\neq$}1
{<=}{$\leq$}1
{<=m}{$\leq_m$}1
{<=l}{$\leq_l$}1
{->}{$\to$}1
{<->}{$\leftrightarrow$}1
{==>}{$\Longrightarrow$}1
{~~}{$\neg$}1
{=>}{$\Rightarrow$}1
{\#|}{\#|}1
{exists}{$\exists$}1
{\\in}{$\in$}1
{\\sigma_}{$\sigma$}1
{\\omega_}{$\omega$}1
{\\delta}{$\delta$}1
{/\\}{$\land$}1
}

\let\L=\lstinline


%% definitions
%% semantic markup commands
\newcommand\newterm[1]{\emph{#1}}

%% copiable underscore
\newcommand{\us}{\char`\_}

%% basic math notations
\newcommand\imp{\Rightarrow}
\newcommand\tuple[1]{\langle{#1}\rangle}
\newcommand\enum[1]{\{ #1 \}}
\newcommand\comprh[2]{\left\{{#1}\ \middle|\ {#2}\right\}}
%\newcommand\comprh[2]{\left\{{#1}\ :\ {#2}\right\}}
\newcommand\defarrow{\overset{def}\Longleftrightarrow}
\newcommand\defeq{:=}

%% matrix notations
\newcommand\mx[2]{\left[ \begin{array}{#1} #2 \end{array} \right]}
\newcommand\mxc{\mx{cccccccccc}}
\newcommand\mxr{\mx{r}}

%% calculus notations
\newcommand\seq[2][n]{\left({#2}\right)_{{#1}\ge0}}
\newcommand\seqa[2][n]{\left({#2}_{#1}\right)_{{#1}\ge0}}
\newcommand\limseq[1][n]{\lim_{{#1}\to\infty}}
\newcommand\ser[2][k]{\sum_{{#1}=0}^\infty{#2}}

%% sets of numbers
\newcommand\N{\mathbb N}
\newcommand\Z{\mathbb Z}
\newcommand\Q{\mathbb Q}
\newcommand\R{\mathbb R}
\newcommand\Rpos{\mathbb R_{>0}}
\newcommand\Rneg{\mathbb R_{<0}}
\newcommand\Rnneg{\mathbb R_{\geq0}}
\newcommand\Rnpos{\mathbb R_{\leq0}}
\newcommand\Rext{\mathbb R \cup \enum{\pm\infty}}

%% set theory
\DeclareMathOperator{\dom}{dom}
\DeclareMathOperator{\ran}{ran}
\DeclareMathOperator{\id}{id}
\newcommand\power{\mathscr{P}}


\newcommand\infotheo{\textsc{InfoTheo}}
\newcommand\coq{\textsc{Coq}}
\newcommand\mathcomp{\textsc{MathComp}}

\newcommand\coqin[1]{\mintinline{ssr}{#1}}

\newcommand\Set{\mathbf{Set}}
\newcommand\Cat{\mathbf{Cat}}
\newcommand\Hilb{\mathbf{FdHilb}}

\newcommand\CNOT{\mathtt{CNOT}}

\title{OCaml から Coq へのコンパイラ及び翻訳されたプログラムの証明}
\author{中村薫, Jacques Garrigue, 毎田詠人, 才川隆文}

\begin{document}
\begin{frame}[fragile]
  \begin{screen}
    \begin{center}
      {\Huge OCaml から Coq へのコンパイラ及び翻訳されたプログラムの証明} \\
      {\Large 中村薫, Jacques Garrigue, 毎田詠人, 才川隆文}
      \qquad {\large 名古屋大学}
    \end{center}
  \end{screen}

  \begin{columns}[T]
    \begin{column}{0.33\columnwidth}
      \begin{block}{Coqgen project とは}
        \begin{itemize}
          \item OCaml の型推論の正しさを Coq で確認することを目標とするプロジェクト. 
          \item 具体的には OCaml から Coq へのコンパイラの作成, およびコンパイル後のプログラムに関する証明を行っている. 
          \item 以下, Coqgen project で得られた成果の一部をまとめ, 説明する.
        \end{itemize} 
      \end{block}
      \begin{block}{モナド}
        \begin{itemize}
          \item OCaml における store や例外を Coq でも表現することができるようにモナドを利用する. 
          \item store を Env, 例外を Exn で表し, これらをまとめて次のようにモナドを定義する. 
        \end{itemize}
        \begin{lstlisting}{coq}
Definition M T := Env -> Env * (T + Exn).  \end{lstlisting}
      \end{block}
      \begin{block}{Type translation}
        OCaml の型はデータ型として定義されていて, Coq に翻訳される. 
        \begin{lstlisting}{Coq}
Fixpoint coq_type (T : ml_type) : Type := ...  \end{lstlisting}
      \end{block}
      \begin{block}{今までの Coqgen}
        \begin{itemize}
          \item core ML($\lambda$計算 $+$ 多相型, 再帰関数, データ型)
          \item 参照型, 例外
          \item 変換の例
        \end{itemize}
        \begin{lstlisting}{Caml}
let rec fact_rec n =
  if n <= 1 then 1 else n * fact_rec (n - 1)  \end{lstlisting}
        \begin{lstlisting}{Coq}
Fixpoint fact_rec (h : nat) (n : coq_type ml_int) : M (coq_type ml_int) :=
  if h is h.+1 then (* 停止性のためにガスを減らす *)
    do v <- ml_le h ml_int n 1%int63;
    if v then Ret 1%int63 else
      do v <- fact_rec h (Int63.sub n 1%int63); Ret (Int63.mul n v)
  else FailGas.  \end{lstlisting}
      \end{block}
      \begin{block}{while loop}
        \begin{itemize}
          \item OCaml のプログラム
        \end{itemize}
        \begin{lstlisting}{Caml}
let fact n =
  let i = ref n in let v = ref 1 in
  while !i > 0 do v := !v * !i; i := !i - 1 done;  !v  \end{lstlisting}
        \begin{itemize}
          \item Coqgen のライブラリ
        \end{itemize}
          \begin{lstlisting}{Coq}
Fixpoint whileloop (h : nat) (f : M bool) (b : M unit) : M unit :=
  if h is h.+1 then
    do v <- f; if v then (do _ <- b; whileloop h f b) else Ret tt
  else FailGas.  \end{lstlisting}
        \begin{itemize}
          \item コンパイル結果
        \end{itemize}
        \begin{lstlisting}{Coq}
Definition fact (h : nat) (n : coq_type ml_int) : M (coq_type ml_int) :=
  do i <- newref ml_int n;
  do v <- newref ml_int 1%int63;
  do _ <-
  whileloop h (do v_1 <- getref ml_int i; ml_gt h ml_int v_1 0%int63)
    (do _ <-
    (do v_1 <-
      (do v_1 <- getref ml_int i;
      do v_2 <- getref ml_int v; Ret (Int63.mul v_2 v_1));
      setref ml_int v v_1);
    do v_1 <- (do v_1 <- getref ml_int i; Ret (Int63.sub v_1 1%int63));
    setref ml_int i v_1);
  getref ml_int v.  \end{lstlisting}
      \end{block}
    \end{column}

    \begin{column}{0.33\columnwidth}
      \begin{block}{lazy}
        \begin{itemize}
          \item OCaml のプログラム
        \end{itemize}
        \begin{lstlisting}{Caml}
let lazy_id = lazy (fun x -> x) (* 多相型  (`a -> `a)lazy_t *)

let lazy_next c = lazy (incr c; !c)

let c = ref 0 in let m = lazy_next c in let n = lazy_next c in
let n = Lazy.force n in let m = Lazy.force m in [m; n]  \end{lstlisting}

\begin{itemize}
  \item Coqgen のライブラリ
\end{itemize}
\begin{lstlisting}{Coq}
Inductive lazy_val (a : Type) :=
  LzVal of a | LzThunk of M a | LzExn of ml_exns.
Inductive lazy_t a a1 := Lval of a | Lref of (loc (ml_lazy_val a1)).

Definition force a (lz : coq_type (ml_lazy a)) : M (coq_type a) := ...
Definition make_lazy a (b : M (coq_type a)) : M (coq_type (ml_lazy a)) :=
  do x <- newref (ml_lazy_val a) (LzThunk _ b); Ret (Lref _ _ x).
Definition make_lazy_val a (b : coq_type a) : coq_type (ml_lazy a) :=
  Lval (coq_type a) a b.
\end{lstlisting}
        \begin{itemize}
          \item コンパイル結果
        \end{itemize}
        \begin{lstlisting}{Coq}
Definition lazy_id (T : ml_type) : coq_type (ml_lazy (ml_arrow T T)) :=
  make_lazy_val (ml_arrow T T) (fun x : coq_type T => Ret x).  \end{lstlisting}
        make\_lazy\_val を使うことで多相型にできる. 
        \begin{lstlisting}{Coq}
Definition lazy_next (c : coq_type (ml_ref ml_int)) 
  : M (coq_type (ml_lazy ml_int)) :=
      make_lazy ml_int (do _ <- incr c; getref ml_int c).

Definition it_2 := Eval compute in
  Restart it_1
    (do c <- newref ml_int 0%int63;
     do m <- lazy_next c; do n <- lazy_next c;
     do n_1 <- force ml_int n; do m_1 <- force ml_int m;
     Ret (m_1 :: n_1 :: @nil (coq_type ml_int))). 
Print it_2.
it_2 = (..., inl [:: 2%sint63; 1%sint63]) : Env * (seq int + Exn)  \end{lstlisting}
force されて初めて計算が実行されるという lazy の性質が表現できている. 
      \end{block}
      \begin{block}{insertion sort}
        \begin{itemize}
          \item OCaml のプログラム
        \end{itemize}
        \begin{lstlisting}{Caml}
let rec insert a l =
  match l with
  | [] -> [a]
  | b :: l' -> if a <= b then a :: l else b :: insert a l'
let rec isort l =
  match l with 
  | [] -> []
  | a :: l' -> insert a (isort l')  \end{lstlisting}
        \begin{itemize}
          \item コンパイル結果
        \end{itemize}
        \begin{lstlisting}{Coq}
Fixpoint insert (h : nat) (T_1 : ml_type) (a : coq_type T_1)
  (l : coq_type (ml_list T_1)) : M (coq_type (ml_list T_1)) :=
  if h is h.+1 then
    match l with
    | @nil _ => Ret (a :: @nil (coq_type T_1))
    | b :: l' =>
      do v <- ml_le h T_1 a b;
      if v then Ret (a :: l) else
        do v <- insert h T_1 a l'; Ret (@cons (coq_type T_1) b v)
    end
  else FailGas.  \end{lstlisting}
      \end{block}
    \end{column}

    \begin{column}{0.33\columnwidth}
      \begin{block}{insertion sort 続}
        \begin{lstlisting}{Coq}
Fixpoint isort (h : nat) (T_1 : ml_type) (l_1 : coq_type (ml_list T_1))
  : M (coq_type (ml_list T_1)) :=
  if h is h.+1 then
    match l_1 with
    | @nil _ => Ret (@nil (coq_type T_1))
    | a :: l' => do v <- isort h T_1 l'; insert h T_1 a v
    end
  else FailGas.  \end{lstlisting}
      \end{block}
      \begin{block}{insertion sort の証明}
          整列している状態 sorted の定義
        \begin{lstlisting}{Coq}
Fixpoint sorted l :=
  if l is a :: l' then all (le a) l' && sorted l' else true.  \end{lstlisting}
          示したいことは, isort で作られた列が上の意味で整列していることである. 
        \begin{lstlisting}{Coq}
Theorem isort_ok h l l' : isort h ml_int l = Ret l' -> sorted l'.
Proof.
  elim: h l l' => [_ _ |h IH [|a l] l'] /(happly empty_env) //=.
  - by move=> [] <-.
  - destruct h => //.
    case: (isort_pure h.+1 l) => -[l'' H].
    + rewrite H bindretf.
      case: (insert_pure h.+1 a l'') => -[l0 H'].
      - rewrite H' => -[] <-.
        by apply /(insert_ok H') /(IH l l'').
      - by rewrite H'.
    + by rewrite H bindfailf.
Qed.  \end{lstlisting}
補題の先頭の isort h ml\_int l = Ret l'は「計算が成功すれば」という条件である.

        証明中に使われている補題は以下のとおりである. 
        \begin{lstlisting}{Coq}
Lemma happly [A B] [f g : A -> B] x : f = g -> f x = g x.
Lemma bindretf A B (a : A) (f : A -> M B) : Ret a >>= f = f a.
Lemma bindfailf A B e (g : A -> M B) : @Fail A e >>= g = @Fail B e.

Definition pure [T] (m : M T) :=
  (exists r, m = Ret r) \/ (exists e, m = Fail e).
Lemma insert_pure h b l : pure (insert h ml_int b l).
Lemma isort_pure h l : pure (isort h ml_int l).
Lemma insert_ok h a l l' : insert h ml_int a l = Ret l' ->
  sorted l -> sorted l'.  \end{lstlisting}
      \end{block}
      \begin{block}{直接 Coq に書く場合との違い}
        コンパイルされたプログラムはモナドに覆われているため, 
        一般に証明は煩雑になる. 
        しかし, 上のような bindretf や insert\_pure などの
        モナドに関する補題を用意しておくことで, 直接 Coq に書いた場合の
        証明に近づけることができる.

        例えば, isort\_ok であれば, 直接書くと 
        \begin{lstlisting}{Coq}
Theorem isort_ok l : sorted (isort l).
Proof. elim: l => //= a l IH; by rewrite insert_ok. Qed.  \end{lstlisting}
        となるが, モナドに関する部分を除けば本質が同じであることが見て取れる. 
        今後はさらに近づける方法を考えたい. 
      \end{block}
        Coqgen に関する情報
        \begin{center}
        \url{https://github.com/COCTI/ocaml/pull/3}
        \end{center}
        本研究はTezos財団及びJSPS科研費JP22K11902の助成を受けたものです. 
    \end{column}
  \end{columns}
 \end{frame}
\end{document}