\documentclass[11pt]{jarticle}
%%%%%%%%%%%%%%%%%%%%%%%%%%%%%%%%%%%%%%%%%%%%%%%%%%%%%%%%%%%%%%%%%%%%%%%%%%%
\usepackage[a4paper,margin=25mm]{geometry}
\usepackage{amsmath}
\usepackage{amsthm}
\usepackage{ascmac}
\newtheoremstyle{mystyle}% % Name
    {}%                      % Space above
    {}%                      % Space below
    {\normalfont}%           % Body font
    {}%                      % Indent amount
    {\bfseries}%             % Theorem head font
    {}%                      % Punctuation after theorem head
    { }%                     % Space after theorem head, ' ', or \newline
    {}%                      % Theorem head spec (can be left empty, meaning `normal')
\theoremstyle{mystyle}
\usepackage{amssymb}
\usepackage{ascmac}
\usepackage{txfonts}
%\usepackage{graphics}
\usepackage{ascmac}

\usepackage{amsmath}
\usepackage{amssymb}

\usepackage{cite}
\usepackage{graphicx}
\usepackage{enumerate}
\usepackage{url}
\usepackage{mathrsfs}
\usepackage{eucal}
\usepackage{listings}

% %% amsthm
% \usepackage{amsthm}
% \theoremstyle{plain}
% \newtheorem{theorem}{定理}[section]
% \newtheorem{lemma}[theorem]{補題}
% \newtheorem{condition}[theorem]{条件}
% \newtheorem{axiom}[theorem]{公理}
% \theoremstyle{definition}
% \newtheorem{definition}[theorem]{定義}
% \newtheorem{exercise}[theorem]{演習}
% \newtheorem{remark}[theorem]{注意}
% \newtheorem{example}[theorem]{例}
% \newtheorem*{definition*}{定義}
% \newtheorem*{exercise*}{演習}
% \newtheorem*{remark*}{注意}
% \newtheorem*{example*}{例}

%% xcolor
\usepackage{xcolor}
\definecolor{darkgreen}{rgb}{0,0.5,0}
\definecolor{darkred}{rgb}{0.5,0,0}
\definecolor{darkyellow}{rgb}{0.3,0.3,0}
\definecolor{darkorange}{rgb}{0.4,0.2,0}
\definecolor{myviolet}{rgb}{0.6,0.0,0.65}
\definecolor{myblue}{rgb}{0.1,0.0,0.8}
\definecolor{mygreen}{rgb}{0.2,0.7,0.1}
\definecolor{mycomment}{rgb}{0.7,0.2,0.2}
\definecolor{myred}{rgb}{0.8,0.0,0.0}
\definecolor{myorange}{rgb}{0.6,0.2,0.2}

% \pgfdeclarelayer{background}
% \pgfdeclarelayer{foreground}
% \pgfsetlayers{background,main,foreground}

\lstdefinelanguage{coq}[]{Caml}{
keywords=[1]{Section,Definition,Defined,CoInductive,Coercion,Inductive,Fixpoint,
  Parameter,Module,Import,Record,Structure,Axiom,Lemma,Theorem,Notation,
  Reserved,End,Proof,Goal,Qed,Variable,Variables,Hypothesis,Let,Program,Canonical},
keywordstyle=\color{myviolet}\ttfamily,
morekeywords=[2]{match,with,end,Set,Prop,Type,fun,of,let,in,struct,if,is,then,else,return},
keywordstyle=[2]\color{mygreen}\ttfamily,
morekeywords=[3]{move},
keywordstyle=[3]\color{myblue}\ttfamily,
morekeywords=[3]{field,lra},
keywordstyle=[3]\color{myred}\ttfamily
}

\lstset{
language=coq,
columns=fullflexible,
keepspaces,
basicstyle=\small\ttfamily,
identifierstyle=\color{black}\ttfamily,
commentstyle=\color{mycomment}\it\ttfamily,
morecomment=[n]{(*}{*)},
morestring=[b][\color{myorange}\ttfamily]",
showstringspaces=false,
extendedchars=true,
literate=
{forall}{$\forall$}1
{!=}{$\neq$}1
{<>}{$\neq$}1
{<=}{$\leq$}1
{<=m}{$\leq_m$}1
{<=l}{$\leq_l$}1
{->}{$\to$}1
{<->}{$\leftrightarrow$}1
{==>}{$\Longrightarrow$}1
{~~}{$\neg$}1
{=>}{$\Rightarrow$}1
{\#|}{\#|}1
{exists}{$\exists$}1
{\\in}{$\in$}1
{\\sigma_}{$\sigma$}1
{\\omega_}{$\omega$}1
{\\delta}{$\delta$}1
{/\\}{$\land$}1
}

\let\L=\lstinline

%%%%%%%%%%%%%%%%%%%%%%%%%%%%%%%%%%%%%%%%%%%%%%%%%%%%%%%%%%%%%%%%%%%%%%%%%%%
\newtheorem{df}{$\textrm{Definition}$}[subsection]
\newtheorem{ex}[df]{$\textrm{Example}$}
\newtheorem{prop}[df]{$\textrm{Proposition}$}
\newtheorem{lem}[df]{$\textrm{Lemmma}$}
\newtheorem{cor}[df]{$\textrm{Corollary}$}
\newtheorem{rmk}[df]{$\textrm{Remark}$}
\newtheorem{thm}[df]{$\textrm{Theorem}$}
\newtheorem{axi}[df]{$\textrm{Axiom}$}
%%%%%%%%%%%%%%%%%%%%%%%%%%%%%%%%%%%%%%%%%%%%%%%%%%%%%%%%%%%%%%%%%%%%%%%%%%%
\newcommand{\bdf}{\begin{shadebox} \begin{df}}
\newcommand{\edf}{\end{df} \end{shadebox}}
\newcommand{\bex}{\begin{ex}}
\newcommand{\eex}{\end{ex}}
\newcommand{\bprop}{\begin{shadebox} \begin{prop}}
\newcommand{\eprop}{\end{prop} \end{shadebox}}
\newcommand{\blem}{\begin{shadebox} \begin{lem}}
\newcommand{\elem}{\end{lem} \end{shadebox}}
\newcommand{\bcor}{\begin{shadebox} \begin{cor}}
\newcommand{\ecor}{\end{cor} \end{shadebox}}
\newcommand{\brmk}{\begin{rmk}}
\newcommand{\ermk}{\end{rmk}}
\newcommand{\bthm}{\begin{shadebox} \begin{thm}}
\newcommand{\ethm}{\end{thm} \end{shadebox}}
\newcommand{\bpf}{\begin{proof}}
\newcommand{\epf}{\end{proof}}
\newcommand{\N}{\mathbb{N}}
\newcommand{\Z}{\mathbb{Z}}
\newcommand{\Q}{\mathbb{Q}}
\newcommand{\R}{\mathbb{R}}
\newcommand{\C}{\mathbb{C}}
\newcommand{\D}{\mathbb{D}}
\newcommand{\U}{\mathcal{U}}
\newcommand{\plim}{{\displaystyle \lim_{\substack{\longleftarrow\\ n}}}}
\newcommand{\seq}[2]{(#1_{#2})_{#2\ge1}}
\newcommand{\Zmod}[1]{\Z/p^{#1}\Z}
\newcommand{\Lra}{\Longrightarrow}
\newcommand{\Llra}{\Longleftrightarrow}
\newcommand{\lra}{\leftrightarrow}
\newcommand{\ra}{\rightarrow}
\newcommand{\ol}[1]{{\overline{#1}}}
\newcommand{\ul}[1]{{\underline{#1}}}
\newcommand{\sem}[1]{[\hspace{-2pt}[{#1}]\hspace{-2pt}]}
\newcommand{\pros}[1]{\begin{array}{c} \ast \ast\\ \tt{#1} \end{array}}
\newcommand{\pgm}[1]{{\tt{#1}}\text{-プログラム}}
\newcommand{\pgms}[1]{{\tt{#1}}\text{-プログラム集合}}
\newcommand{\dtm}[1]{{\tt{#1}}\text{-データ}}
\newcommand{\data}[1]{{\tt{#1}}\text{-データ集合}}
\newcommand{\rtf}[3]{time^{\tt{#1}}_{\tt{#2}}({\tt{#3}})}
\newcommand{\id}{\textrm{id}}
\newcommand{\refl}{\textrm{refl}}
\newcommand{\ap}{\textrm{ap}}
\newcommand{\apd}{\textrm{apd}}
\newcommand{\pr}[1]{\textrm{pr}_{#1}}
\newcommand{\tp}{\textrm{transport}}
\newcommand{\qinv}{\textrm{qinv}}
\newcommand{\iseq}{\textrm{isequiv}}
\newcommand{\peq}{\textrm{pair}^=}
\newcommand{\sig}[3]{\sum_{{#1} : {#2}} {#3}\ ({#1})}
\newcommand{\0}{\textbf{0}}
\newcommand{\1}{\textbf{1}}
\newcommand{\2}{\textbf{2}}
\newcommand{\fune}{\textrm{funext}}
\newcommand{\happ}{\textrm{happly}}
\newcommand{\ua}{\textrm{ua}}
\newcommand{\ide}{\textrm{idtoequiv}}
\newcommand{\set}[1]{\textrm{isSet({#1})}}
%%%%%%%%%%%%%%%%%%%%%%%%%%%%%%%%%%%%%%%%%%%%%%%%%%%%%%%%%%%%%%%%%%%%%%%%%%%
\begin{document}
\title{少人数クラス内容報告}
\author{アドバイザー $\colon$ Jacques Garrigue教授\\
           学籍番号 $\colon$ 322101289\\
           氏名 $\colon$ 中村 薫}
\date{\today}
\maketitle
%%%%%%%%%%%%%%%%%%%%%%%%%%%%%%%%%%%%%%%%%%%%%%%%%%%%%%%%%%%%%%%%%%%%%%%%%%%
\tableofcontents
%%%%%%%%%%%%%%%%%%%%%%%%%%%%%%%%%%%%%%%%%%%%%%%%%%%%%%%%%%%%%%%%%%%%%%%%%%%
\section{HoTT}
HoTT とは, Homotopy Type Theory の略であり, 
\begin{align*}
  \text{$a$が型$A$の要素である} &\leftrightarrow \text{$a$が空間$A$の点である} \\
  \text{$a = b$である} &\leftrightarrow \text{点$a$と点$b$の間にパスが存在する}
\end{align*}
というように, 型理論に対してホモトピー的解釈を与えたものである. 本章では, HoTT の大きな特徴の一つである, univalence axiom について説明する. 大雑把にいえば, univalence axiom は「型$A$と型$B$が同型ならば, $A$と$B$は等しい」という公理である. この意味を正確にとらえるため, 型同士の等しさや同型を定義していく. 
%%%%%%%%%%%%%%%%%%%%%%%%%%%%%%%%%%%%%%%%%%%%%%%%%%%%%%%%%%%%%%%%%%%%%%%%%%%
\subsection{型から型を作る}
$A$と$B$の2つの型が与えられたとき, そこから関数型$A \to B$が構成できる. このとき, 
\begin{align*}
  f : A \to B,\ a : A &\Lra f(a) : B \\
  a : A,\ b(x) : B &\Lra \lambda a. b : A \to B 
\end{align*}
である. より一般に, 型$A$と$A$上の型族$B \to \U$が与えられれば($\U$はユニバース), 依存関数型$\prod_{a : A} B(a)$が構成でき, 
\begin{align*}
  f :\prod_{a : A} B(a),\ a : A &\Lra f(a) : B(a) \\
  a : A,\ b(x) : B(x) &\Lra \lambda a. b : \prod_{a : A} B(a) 
\end{align*}
である. さらに, 既存の型から新たな型を作るやり方として, 構成規則, 導入規則, 除去規則, 計算規則の4つを与える帰納的な方法がある. 例えば, 依存和型$\sum_{x : A} B(x)$は, 
\begin{itemize}
  \item 構成規則:$A : \U$, $B : A \to \U \Lra \sum_{x : A} B(x)$
  \item 導入規則:$a : A$, $b : B(a) \Lra (a, b) : \sum_{x : A} B(x)$
  \item 除去規則:ind$_{\sum_{x : A} B(x)} : \prod_{C : (\sum_{x : A} B(x)) \to \U} 
                               \left(\prod_{a : A} \prod_{b : B(a)} C((a, b)) \right) \to 
                                      \prod_{w : \sum_{x : A} B(x)} C(w)$
  \item 計算規則:ind$_{\sum_{x : A} B(x)} (C, g, (a, b)) :\equiv g(a)(b)$
\end{itemize}
で定義できる. 除去規則は, 「任意の$w : \sum_{x : A} B(x)$について$C(w)$を示したければ, 任意の$a : A$, $b : B(a)$について$C((a, b))$を示せばよい」と読むことができる. 
ここで, Curry-Howard 同型に基づいて考えると, 「ある要素$a$とある要素$b$が等しい」という命題は, なにかしらの型と対応するはずである. よってその型 identity type を, 
\begin{itemize}
  \item 構成規則:$A : \U \Lra \_ =_A \_ : \U$
  \item 導入規則:refl$_a : \prod_{a : A} (a =_A a)$
  \item 除去規則:ind$_{=_A} : \prod_{\left( C : \prod_{(x, y : A)} (x =_A y) \to \U \right)} 
                                       \left( \prod_{(x : A)} C(x, x, {\rm refl}_x) \right) \to
                                       \prod_{(x, y : A)} \prod_{(p : x =_A y)} C(x, y, p)$
  \item 計算規則:ind$_{=_A} (C, c, x, x, {\rm refl}_x) : \equiv c(x)$
\end{itemize}
と定義する. 除去規則は, 依存和型のときと同様に考えると, 「任意の$x$, $y : A$, $x = y$について$C(x, y, p)$を示したければ, 任意の$x : A$について$C(x, x, \refl{x})$を示せばよい」となる. 
\subsection{型の同型}
ここで, 型と型の間の同型を定義したい. まず, 関数の間のホモトピーを定義する. 
\bdf[\cite{hott} Definition 2.4.1]
  $A : \U$, $P : A \to \U$とする. $f$, $g : \prod_{x : A} P(x)$に対して, 
  \[
    (f \sim g) :\equiv \prod_{x : A} (f(x) = g(x))
  \]
  と定める. 
\edf
次に, 「逆写像」を定義する. 
\bdf[\cite{hott} Definition 2.4.6]
  $A$, $B : \U$, $f : A \to B$とする. このとき, $f$の quasi-inverse qinv($f$)を, 
  \[
    \qinv(f) :\equiv \sum_{g : B \to A} ((f \circ g \sim \id_B) \times
                                                   (g \circ f \sim \id_A))
  \]
\edf
例えば, id$_A$の quasi-inverse はid$_A$自身である. さらに, この qinv を用いて, isequiv を, 
\begin{itemize}
  \item qinv($f$) $\to$ isequiv($f$)
  \item isequiv($f$) $\to$ qinv($f$)
  \item $e_1$, $e_2 :$isequiv($f$) ならば $e_1 = e_2$
\end{itemize}
をみたすものとして定義したい. ここでは, 
\[
  \iseq(f) :\equiv \left( \sum_{g : B \to A} (f \circ g \sim \id_B) \right) \times
                           \left( \sum_{h : B \to A} (h \circ f \sim \id_A) \right) \quad
                           \text{(\cite{hott} p73 (2.4.10))}
\]
と定めることにする. isequiv を使って型同士の同型を定義する. 
\bdf[\cite{hott} p73 (2.4.11)]
  $A$, $B : \U$について, 
  \[
    A \simeq B :\equiv \sum_{f : A \to B} \iseq(f)
  \]
  と定める. 
\edf
\bex
$A$, $B$, $C : \U$について, 
\begin{itemize}
  \item $A \simeq A$
  \item $A \simeq B \to B \simeq A$
  \item $A \simeq B \to B \simeq C \to A \simeq C$
\end{itemize}
などが成り立つ. 
\eex
%%%%%%%%%%%%%%%%%%%%%%%%%%%%%%%%%%%%%%%%%%%%%%%%%%%%%%%%%%%%%%%%%%%%%%%%%%%
\subsection{Univalence axiom}
これまでに定義した$=$と$\simeq$を用いて, univalence axiom の主張を正しく述べる. まず, 
\[
  \textrm{idtoeqv} : \prod_{A, B : \U} (A =_{\U} B) \to (A \simeq B)
\]
を定める. この関数が存在することは, path induction よりわかる. このidtoeqvに対して, 
\begin{shadebox}
\begin{axi}[\cite{hott} Axiom 2.10.3]
  \[
    \textrm{ua} : \prod_{A, B : \U} \iseq(\textrm{idtoeqv}(A, B))
  \]
\end{axi}
\end{shadebox}
が univalence axiom である. とくに, この公理を仮定すれば, 
\[
  (A =_{\U} B) \simeq (A \simeq B)
\]
となる. さらに, 関数の外延性
\[
  \textrm{funext} : \left( \prod_{x : A} (f(x) = g(x)) \right) \to (f = g)
\]
が従うことも知られている. 
%%%%%%%%%%%%%%%%%%%%%%%%%%%%%%%%%%%%%%%%%%%%%%%%%%%%%%%%%%%%%%%%%%%%%%%%%%%
\begin{thebibliography}{9}
  \bibitem{hott} The Univalent Foundations Program, 
                      {\it{Homotopy Type Theory: Univalent Foundations of Mathematics}}
\end{thebibliography}
\end{document}