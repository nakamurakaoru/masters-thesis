\documentclass[11pt]{jsreport}
%%%%%%%%%%%%%%%%%%%%%%%%%%%%%%%%%%%%%%%%%%%%%%%%%%%%%%%%%%%%%%%%%%%%%%%%%%%
\usepackage[a4paper,margin=25mm]{geometry}
\usepackage{amsmath}
\usepackage{amsthm}
\usepackage{ascmac}
\newtheoremstyle{mystyle}% % Name
    {}%                      % Space above
    {}%                      % Space below
    {\normalfont}%           % Body font
    {}%                      % Indent amount
    {\bfseries}%             % Theorem head font
    {}%                      % Punctuation after theorem head
    { }%                     % Space after theorem head, ' ', or \newline
    {}%                      % Theorem head spec (can be left empty, meaning `normal')
\theoremstyle{mystyle}
\usepackage{amssymb}
\usepackage{ascmac}
\usepackage{txfonts}
%\usepackage{graphics}
\usepackage{ascmac}

\usepackage{amsmath}
\usepackage{amssymb}

\usepackage{cite}
\usepackage{graphicx}
\usepackage{enumerate}
\usepackage{url}
\usepackage{mathrsfs}
\usepackage{eucal}
\usepackage{listings}

% %% amsthm
% \usepackage{amsthm}
% \theoremstyle{plain}
% \newtheorem{theorem}{定理}[section]
% \newtheorem{lemma}[theorem]{補題}
% \newtheorem{condition}[theorem]{条件}
% \newtheorem{axiom}[theorem]{公理}
% \theoremstyle{definition}
% \newtheorem{definition}[theorem]{定義}
% \newtheorem{exercise}[theorem]{演習}
% \newtheorem{remark}[theorem]{注意}
% \newtheorem{example}[theorem]{例}
% \newtheorem*{definition*}{定義}
% \newtheorem*{exercise*}{演習}
% \newtheorem*{remark*}{注意}
% \newtheorem*{example*}{例}

%% xcolor
\usepackage{xcolor}
\definecolor{darkgreen}{rgb}{0,0.5,0}
\definecolor{darkred}{rgb}{0.5,0,0}
\definecolor{darkyellow}{rgb}{0.3,0.3,0}
\definecolor{darkorange}{rgb}{0.4,0.2,0}
\definecolor{myviolet}{rgb}{0.6,0.0,0.65}
\definecolor{myblue}{rgb}{0.1,0.0,0.8}
\definecolor{mygreen}{rgb}{0.2,0.7,0.1}
\definecolor{mycomment}{rgb}{0.7,0.2,0.2}
\definecolor{myred}{rgb}{0.8,0.0,0.0}
\definecolor{myorange}{rgb}{0.6,0.2,0.2}

% \pgfdeclarelayer{background}
% \pgfdeclarelayer{foreground}
% \pgfsetlayers{background,main,foreground}

\lstdefinelanguage{coq}[]{Caml}{
keywords=[1]{Section,Definition,Defined,CoInductive,Coercion,Inductive,Fixpoint,
  Parameter,Module,Import,Record,Structure,Axiom,Lemma,Theorem,Notation,
  Reserved,End,Proof,Goal,Qed,Variable,Variables,Hypothesis,Let,Program,Canonical},
keywordstyle=\color{myviolet}\ttfamily,
morekeywords=[2]{match,with,end,Set,Prop,Type,fun,of,let,in,struct,if,is,then,else,return},
keywordstyle=[2]\color{mygreen}\ttfamily,
morekeywords=[3]{move},
keywordstyle=[3]\color{myblue}\ttfamily,
morekeywords=[3]{field,lra},
keywordstyle=[3]\color{myred}\ttfamily
}

\lstset{
language=coq,
columns=fullflexible,
keepspaces,
basicstyle=\small\ttfamily,
identifierstyle=\color{black}\ttfamily,
commentstyle=\color{mycomment}\it\ttfamily,
morecomment=[n]{(*}{*)},
morestring=[b][\color{myorange}\ttfamily]",
showstringspaces=false,
extendedchars=true,
literate=
{forall}{$\forall$}1
{!=}{$\neq$}1
{<>}{$\neq$}1
{<=}{$\leq$}1
{<=m}{$\leq_m$}1
{<=l}{$\leq_l$}1
{->}{$\to$}1
{<->}{$\leftrightarrow$}1
{==>}{$\Longrightarrow$}1
{~~}{$\neg$}1
{=>}{$\Rightarrow$}1
{\#|}{\#|}1
{exists}{$\exists$}1
{\\in}{$\in$}1
{\\sigma_}{$\sigma$}1
{\\omega_}{$\omega$}1
{\\delta}{$\delta$}1
{/\\}{$\land$}1
}

\let\L=\lstinline

%%%%%%%%%%%%%%%%%%%%%%%%%%%%%%%%%%%%%%%%%%%%%%%%%%%%%%%%%%%%%%%%%%%%%%%%%%%
\newtheorem{df}{$\textrm{Definition}$}[subsection]
\newtheorem{ex}[df]{$\textrm{Example}$}
\newtheorem{prop}[df]{$\textrm{Proposition}$}
\newtheorem{lem}[df]{$\textrm{Lemma}$}
\newtheorem{cor}[df]{$\textrm{Corollary}$}
\newtheorem{rmk}[df]{$\textrm{Remark}$}
\newtheorem{thm}[df]{$\textrm{Theorem}$}
\newtheorem{axi}[df]{$\textrm{Axiom}$}
%%%%%%%%%%%%%%%%%%%%%%%%%%%%%%%%%%%%%%%%%%%%%%%%%%%%%%%%%%%%%%%%%%%%%%%%%%%
%% theorems %%%%%%%%%%%%%%%%%%%%%%%%%%%%%%%%%%%%%%%%%%%%%%%%%%%%%%%%%%%%%%%%
\newcommand{\bdf}{\begin{shadebox} \begin{df}}
\newcommand{\edf}{\end{df} \end{shadebox}}
\newcommand{\bex}{\begin{ex}}
\newcommand{\eex}{\end{ex}}
\newcommand{\bprop}{\begin{shadebox} \begin{prop}}
\newcommand{\eprop}{\end{prop} \end{shadebox}}
\newcommand{\blem}{\begin{shadebox} \begin{lem}}
\newcommand{\elem}{\end{lem} \end{shadebox}}
\newcommand{\bcor}{\begin{shadebox} \begin{cor}}
\newcommand{\ecor}{\end{cor} \end{shadebox}}
\newcommand{\brmk}{\begin{rmk}}
\newcommand{\ermk}{\end{rmk}}
\newcommand{\bthm}{\begin{shadebox} \begin{thm}}
\newcommand{\ethm}{\end{thm} \end{shadebox}}
\newcommand{\baxi}{\begin{shadebox} \begin{axi}}
\newcommand{\eaxi}{\end{axi} \end{shadebox}}
\newcommand{\bpf}{\begin{proof}}
\newcommand{\epf}{\end{proof}}
%% always %%%%%%%%%%%%%%%%%%%%%%%%%%%%%%%%%%%%%%%%%%%%%%%%%%%%%%%%%%%%%%%%%%
\newcommand{\Lra}{\Longrightarrow}
\newcommand{\Llra}{\Longleftrightarrow}
\newcommand{\lra}{\leftrightarrow}
\newcommand{\ra}{\rightarrow}
\newcommand{\ol}[1]{{\overline{#1}}}
\newcommand{\ul}[1]{{\underline{#1}}}
%% sets %%%%%%%%%%%%%%%%%%%%%%%%%%%%%%%%%%%%%%%%%%%%%%%%%%%%%%%%%%%%%%%%%%%%
\newcommand{\N}{\mathbb{N}}
\newcommand{\Z}{\mathbb{Z}}
\newcommand{\Q}{\mathbb{Q}}
\newcommand{\R}{\mathbb{R}}
\newcommand{\C}{\mathbb{C}}
\newcommand{\K}{\mathbb{K}}
\newcommand{\D}{\mathbb{D}}
%% q-analogue %%%%%%%%%%%%%%%%%%%%%%%%%%%%%%%%%%%%%%%%%%%%%%%%%%%%%%%%%%%%%%%
\newcommand{\qcoe}[2]{\left[\begin{array}{ccc}#1\\#2\end{array}\right]}
%% algebra %%%%%%%%%%%%%%%%%%%%%%%%%%%%%%%%%%%%%%%%%%%%%%%%%%%%%%%%%%%%%%%%%%
\newcommand{\plim}{{\displaystyle \lim_{\substack{\longleftarrow\\ n}}}}
\newcommand{\seq}[2]{(#1_{#2})_{#2\ge1}}
\newcommand{\Zmod}[1]{\Z/p^{#1}\Z}
\newcommand{\sem}[1]{[\hspace{-2pt}[{#1}]\hspace{-2pt}]}
\newcommand{\pros}[1]{\begin{array}{c} \ast \ast\\ \tt{#1} \end{array}}
%% Jones %%%%%%%%%%%%%%%%%%%%%%%%%%%%%%%%%%%%%%%%%%%%%%%%%%%%%%%%%%%%%%%%%%%
\newcommand{\pgm}[1]{{\tt{#1}}\text{-プログラム}}
\newcommand{\pgms}[1]{{\tt{#1}}\text{-プログラム集合}}
\newcommand{\dtm}[1]{{\tt{#1}}\text{-データ}}
\newcommand{\data}[1]{{\tt{#1}}\text{-データ集合}}
\newcommand{\rtf}[3]{time^{\tt{#1}}_{\tt{#2}}({\tt{#3}})}
%% Barendiregt %%%%%%%%%%%%%%%%%%%%%%%%%%%%%%%%%%%%%%%%%%%%%%%%%%%%%%%%%%%%%%
\newcommand{\thra}{\twoheadrightarrow}
\newcommand{\lama}{\lambda \! \! \to}
%% HoTT %%%%%%%%%%%%%%%%%%%%%%%%%%%%%%%%%%%%%%%%%%%%%%%%%%%%%%%%%%%%%%%%%%%
\newcommand{\U}{\mathcal{U}}
\newcommand{\id}{\textrm{id}}
\newcommand{\refl}{\textrm{refl}}
\newcommand{\ap}{\textrm{ap}}
\newcommand{\apd}{\textrm{apd}}
\newcommand{\pr}{\textrm{pr}}
\newcommand{\tp}{\textrm{transport}}
\newcommand{\qinv}{\textrm{qinv}}
\newcommand{\iseq}{\textrm{isequiv}}
\newcommand{\peq}{\textrm{pair}^=}
\newcommand{\sig}[3]{\sum_{{#1} : {#2}} {#3}\ ({#1})}
\newcommand{\0}{\textbf{0}}
\newcommand{\1}{\textbf{1}}
\newcommand{\2}{\textbf{2}}
\newcommand{\fune}{\textrm{funext}}
\newcommand{\happ}{\textrm{happly}}
\newcommand{\ua}{\textrm{ua}}
\newcommand{\ide}{\textrm{idtoequiv}}
\newcommand{\set}[1]{\textrm{isSet({#1})}}
\newcommand{\fib}{\textrm{fib}}
\newcommand{\iscont}{\textrm{isContr}}
\newcommand{\total}{\textrm{total}}
\newcommand{\idtoeqv}{\textrm{idtoeqv}}
%%%%%%%%%%%%%%%%%%%%%%%%%%%%%%%%%%%%%%%%%%%%%%%%%%%%%%%%%%%%%%%%%%%%%%%%%%%
\begin{document}
\title{$q$-類似のCoqによる形式化}
\author{アドバイザー $\colon$ Jacques Garrigue教授\\
           学籍番号 $\colon$ 322101289\\
           氏名 $\colon$ 中村 薫}
\date{\today}
\maketitle
%%%%%%%%%%%%%%%%%%%%%%%%%%%%%%%%%%%%%%%%%%%%%%%%%%%%%%%%%%%%%%%%%%%%%%%%%%%
\tableofcontents
%%%%%%%%%%%%%%%%%%%%%%%%%%%%%%%%%%%%%%%%%%%%%%%%%%%%%%%%%%%%%%%%%%%%%%%%%%%
\subsection*{序文}
本論文の主結果は, $q$-類似の初等的な結果をCoqによって形式化するものである. 具体的にはVictor Kac, Pokman Cheungの{\it Quantum Calculus}\cite{Kac}の4章(4.1)式の$q$-Taylor 展開, 及びその系として得られる Gauss's binomial formula の形式化を目標としている. 
本論文での$q$-類似に関する定義や定理, 証明は\cite{Kac}によるものだが, その形式化を行ったという点において独自性がある. 形式化したコード全体は\url{https://github.com/nakamurakaoru/q-analogue}\cite{coq qana}にある. q\_analogue.v が\cite{Kac}の形式化をしたファイルであり, q\_tool.v は直接$q$-類似に関係はしないが, 形式化をするために自分で用意した補題をまとめたファイルである. 

$q$-類似は, 学部4年次に卒業研究のテーマとして扱ったものである. 実数パラメータ$q$, 実数上の関数$f$に対して
\[
  D_q f(x) \coloneqq \frac{f(qx) - f(x)}{(q - 1) x}
\]
で定義される$q$-微分を出発点とし, この$q$-微分に対してうまく振る舞い, かつ$q$を極限で$1$に近づけると通常の定義に一致するように数学の諸概念を一般化するものである.
例えば, 自然数$n$について$x^n$を定義に沿って$q$-微分すると, 
\[
  D_q x^n = \frac{(qx)^n - x^n}{(q - 1) x} = \frac{q^n - 1}{q - 1} x^{n - 1}
\]
となる. 通常の微分では, $(x^n)' = n x^{n - 1}$となることと比較して, $n$の$q$-類似$[n]$を
\[
  [n] = \frac{q^n - 1}{q - 1}
\]
と定める. また, $(x - a)^n$の$q$-類似は, 
\[
  (x - a)^n_q \coloneqq \begin{cases}
                                  1 & (n = 0) \\
                                  (x - a) (x - qa) \cdots (x - q^{n - 1} a) & (n \ge 1)
                                \end{cases}
\]
と定義することで, $q$-微分と自然数の$q$-類似に対してうまく振る舞う, つまり
\[
  D_q (x - a)^n_q = [n](x - a)^{n - 1}_q
\]
が成り立つ. 更に, 階乗と二項係数の$q$-類似をそれぞれ
\begin{align*}
  [n]! &\coloneqq \begin{cases}
                        1 & (n = 0)\\
                        [n] \times [n   - 1] \cdots [1] & (n \ge 1)
                      \end{cases}\\
  \qcoe{n}{j} &\coloneqq \frac{[n]!}{[j]![n - j]!}
\end{align*}
で定めれば, $q$-Taylor展開, Gauss's binomial formula は以下のように書ける. 
\begin{itembox}{$q$-Taylor展開}
  $f(x)$を$N$次の実係数多項式とする. 任意の$c \in \R$について
  \[
    f(x) = \sum_{j = 0}^N (Dq^j f) (c) \frac{(x - c)^j_q}{[j]!}
  \]
  が成り立つ. 
\end{itembox}
\begin{itembox}{Gauss's binomial formula}
  $x$, $a \in \R$, $n \in \Z_{>0}$について
  \[
    (x+a)^n_q = \sum_{j=0}^n \qcoe{n}{j} q^{\frac{j(j-1)}{2}} a^j x^{n-j}
  \]
  が成り立つ. 
\end{itembox}
この2つを形式化することが本論文の主目的である. 

Coqとは, 定理証明支援系の1つであり, 数学的な証明が論理学の推論規則に沿って正しく書かれているかどうか判定するプログラムである.
人間がチェックすることが難しい複雑な証明でも正しさが保証され, また証明付きプログラミングにも応用される. Mizar, Isabelle/HOL等他にも定理証明支援系は存在するが, 修士1年次後期に履修した授業でCoqの使い方を学んだため, 今回の形式化に利用した. 
Coqは型付き$\lambda$計算という理論に基づいている. この$\lambda$計算については修士1年次に少人数クラスで学習した内容であるため, Coq との関係にも触れつつ本論文でも説明を加える. 今回の証明に関しては, Coqの標準ライブラリ \cite{coq sl}に加えて, 数学の証明のために整備されたライブラリ群である mathcomp \cite{coq mc}も用いている. 
実際にCoqが用いられた有名な例として, 四色定理や Feit-Thompson の定理(奇数位数定理)などがある. 
%ssreflect は四色定理は形式化するため, mathcomp は Feit-Thompson の定理(奇数位定理)の形式化のために整備されたライブラリである. 

今後の展望としては, まずはこれまでに形式化した$q$-類似の各概念が$q \to 1$としたときに通常の数学の概念に一致することの形式化を行いたい. このためには, 現在開発中のライブラリである mathcomp analysis\cite{coq ana}を用いる必要がある. また, このライブラリを用いると無限和に関する形式化も可能であるため, Gauss's binomial formula を無限に拡張したものや, 無限和を用いて定義される指数関数, 三角関数の$q$-類似の形式化にも挑戦していきたい. 
%ただし, 無限に関わる議論の形式化は通常の数学の概念を無限に拡張するよりもさらに煩雑になることには注意が必要である. 

最後に構成について述べる. 
まず\ref{sec qana}節で$q$-類似の概要について説明し, 
\ref{sec lambda}節で修士1年次に少人数クラスで学習したH.P.Barendregtの{\it Lambda Calculi with Types}\cite{Bar}に沿って, 型付き$\lambda$計算について述べる. 
\ref{sec coq}節は Coq の概要についてで, 特に\ref{ssec coq_use}節でCoq や mathcompの使い方を具体例を交えて説明するが, より詳細な情報については萩原 学/アフェルト・レナルドの{\it Coq/SSReflect/Mathcomp}\cite{Hag}等を参照のこと. 
これらの準備のもと, \ref{sec form}節から本題の形式化に入る. \cite{Kac}での定義, 定理を述べた後, その形式化を与え, 必要であれば形式化をするにあたっての注意点を述べることを繰り返すという流れである. 
証明の方針等は基本的に\cite{Kac}の通りであるが, \ref{ssec poly}節では一部\cite{Kac}から離れ, 多項式として$q$-微分や$q$-二項式を定義しなおして形式化を行っている. これらの新たな定義が多項式に対してのもとの定義を適用したものと一致していることの証明も行っている. また, \ref{chap hott}章で少人数クラスで学習した内容についてまとめている. 

%%%%%%%%%%%%%%%%%%%%%%%%%%%%%%%%%%%%%%%%%%%%%%%%%%%%%%%%%%%%%%%%%%%%%%%%%%%
\chapter{修士論文} \label{chap thesis}
%%%%%%%%%%%%%%%%%%%%%%%%%%%%%%%%%%%%%%%%%%%%%%%%%%%%%%%%%%%%%%%%%%%%%%%%%%%
\section{$q$-類似} \label{sec qana}
本節では, 形式化について説明する\ref{sec form}節では詳細には立ち入らない$q$-類似の性質や, $q$-Taylor展開以降で\cite{Kac}で取り上げられている内容について述べる. \ref{ssec infsum}, \ref{ssec jacobi}節は無限和に関するものであるため形式化はできていない. 
%%%%%%%%%%%%%%%%%%%%%%%%%%%%%%%%%%%%%%%%%%%%%%%%%%%%%%%%%%%%%%%%%%%%%%%%%%%
\subsection{$q$-微分について成り立つ諸性質} \label{ssec qprops}
通常の微分は線形作用素であり, また積の微分法や商の微分法などの性質が成り立つ. これらの性質は$q$-微分でもどうなるのであろうか. まずは線形性について見ていく. $a$, $b \in \R$, 実数上の関数$f$, $g$について, 
\begin{align*}
  D_q (af(x) + bg(x)) &= \frac{(af(qx) + bf(qx)) - (af(x) + bf(x))}{(q - 1)x}\\
                           &= \frac{af(qx) - af(x)}{(q - 1)x} + \frac{bg(qx) - bg(x)}{(q - 1)x}\\
                           &= a \frac{f(qx) - f(x)}{(q - 1)x} + b \frac{g(qx) - g(x)}{(q - 1)x}\\
                           &= a D_q f(x) + b D_q g(x)
\end{align*}
と計算できるので, 通常の微分と変わらない形で成り立つ(\cite{Kac} p2). 一方で積の微分法はそのままの形では成り立たない. このことを確認するために, まず$q$-差分
\[
  d_q f(x) = f(qx) - f(x)
\]
が積についてどう振る舞うかを観察する. 実数上の関数$f$, $g$について, 実際に計算してみると, 以下のように計算できる. 
\begin{align*}
  d_q (f(x)g(x)) &= f(qx)g(qx) - f(x)g(x) \\
                    &= f(qx)g(qx) - f(qx)g(x) + f(qx)g(x) - f(x)g(x)\\
                    &= f(qx) d_q g(x) + g(x) d_q f(x)
\end{align*}
ここで, $q$-微分は$q$-差分を用いて
\[
  D_q f(x) = \frac{d_q f(x)}{d_q x}
\]
と表せることを用いると(\cite{Kac}では順番が逆で, 先に$q$-微分を定義している), 
\begin{align*}
  D_q(f(x)g(x)) &= \frac{d_q(f(x)g(x))}{(q - 1)x}\\
                   &= \frac{f(qx) d_q g(x) + g(x) d_q f(x)}{(q - 1)x}\\
                   &= f(qx) \frac{d_q g(x)}{(q - 1)x} + g(x) \frac{d_q f(x)}{(q - 1)x}
\end{align*}
となるので, 積の$q$-微分法は
\[
  D_q (f(x) g(x)) = f(qx) D_q g(x) + g(x) D_q f(x) \quad \text{(\cite{Kac} p3 (1.11))}
\]
となる. また, $f(x)g(x) = g(x)f(x)$から, $f$と$g$を入れ替えることで
\[
  D_q (f(x) g(x)) = f(x) D_q g(x) + g(qx) D_q f(x) \quad \text{(\cite{Kac} p3 (1.12))}
\]
も得られる. 次に, 商の微分法については, 
\[
  g(x) \cdot \frac{f(x)}{g(x)} = f(x)
\]
という等式の両辺を$q$-微分し, 左辺について1つ目の積の$q$-微分法を適用することで
\[
  g(qx) D_q \left( \frac{f(x)}{g(x)} \right) + \frac{f(x)}{g(x)} D_q g(x) = D_qf (x)
\]
となるので, 式変形することで商の$q$-微分法
\[
  D_q \left( \frac{f(x)}{g(x)} \right) = \frac{g(x) D_q f(x) - f(x) D_q g(x)}{g(x) g(qx)}
  \quad \text{\cite{Kac} p3 (1.13)}
\]
が得られる. 2つ目の積の$q$-微分法を適用すれば, もう一つの商の微分法
\[
  D_q \left( \frac{f(x)}{g(x)} \right) = \frac{g(qx) D_q f(x) - f(qx) D_q g(x)}{g(x) g(qx)}
  \quad \text{\cite{Kac} p3 (1.14)}
\]
が得られる. 更に, 合成関数の$q$-微分法については一般的なルールは存在せず, $u = u(x) = \alpha x^{\beta}$
のときについてのみ
\[
  D_q f (u (x)) = (D_q^{\beta} f)(u(x)) \cdot D_q u(x) \quad \text{(\cite{Kac} p4 (1.15))}
\]
が成り立つことが知られている(証明は\cite{Kac} p3, p4を参照). 

これらの性質はすべて形式化できており, $D_q$の線形性, 積の$q$-微分法の1つ目・2つ目, 商の$q$-微分法の1つ目・2つ目, 合成関数の$q$-微分法はそれぞれこの順に{\tt Dq\_is\_linear}, {\tt Dq\_prod}, {\tt Dq\_prod'}, {\tt Dq\_quot}, {\tt Dq\_quot'}, {\tt qchain}という補題名で\cite{coq ana}のq\_analogue.vに登録してある. 

また, 今回の主目的である$q$-Taylor展開には利用しないが, 二項係数の$q$-類似について, 通常の二項係数において Pascal の法則が成り立つことの$q$-類似版として, $n > j$をみたす$1$以上の整数$n$, $j$について
\[
  \qcoe{n}{j} = \qcoe{n - 1}{j - 1} + q^j \qcoe{n - 1}{j} \quad \text{(\cite{Kac} p17 (6.2))}
\]
が成り立つことが証明されている(\cite{Kac} p18参照). この補題も形式化できており, \cite{coq qana}のq\_analogue.vの{\tt q\_pascal}という補題である.
%%%%%%%%%%%%%%%%%%%%%%%%%%%%%%%%%%%%%%%%%%%%%%%%%%%%%%%%%%%%%%%%%%%%%%%%%%%
\subsection{無限和に関する話題} \label{ssec infsum}
本節と次節の内容は形式化できていない. 
Gauss's binomial formula において, $x = 1$, $a = x$とすれば
\[
  (1 + x)^n_q = \sum_{j = 0}^n q^{j (j - 1) / 2} \qcoe{n}{j} x^j
\]
となる. ここで$|q| < 1$とすると, 
\[
  \lim_{n \to \infty} \qcoe{n}{j} =
  \lim_{n \to \infty} \frac{(1 - q^n)(1 - q^{n - 1}) \cdots (1 - q^{n - j + 1})}
                                  {(1 - q) (1 - q^2) \cdots (1 - q^j)} =
  \frac{1}{(1 - q) (1 - q^2) \cdots (1 - q^j)}
\]
となるので,  
\[
  (1 + x)^{\infty}_q = \sum_{j = 0}^{\infty} q^{j (j - 1)/2}
  \frac{x^j}{(1 - q) (1 - q^2) \cdots (1 - q^j)}
\]
無限積を無限和に書き換える式が得られる. この式は Euler により見つけられたため, Euler's first identity の意で\cite{Kac}では E1 と呼ばれている. 
%%%%%%%%%%%%%%%%%%%%%%%%%%%%%%%%%%%%%%%%%%%%%%%%%%%%%%%%%%%%%%%%%%%%%%%%%%%
\subsection{$q$-類似を考える利点} \label{ssec jacobi}
%本節では, まだ形式化はできていない$q$-Taylor展開以降の\cite{Kac}の内容について, 特に$q$-類似を考える利点の1つであるJacobiの三重積について説明する. 
%$q$-類似とは, $q \to 1$とすると通常の数学に一致するような拡張のことである. 例えば, 自然数$n$の$q$-類似$[n]$は
%\[
%  [n] = 1 + q + q^2 + \cdots q ^ {n -1} 
%\]
%であり, $(x-a)^n$の$q$-類似$(x-a)^n_q$は
%\[
%  (x-a)^n_q \coloneqq \begin{cases}
%                                  1 & (n=0)\\
%                                  (x-a)(x-qa)\cdots(x-q^{n-1}a) & (n\ge1)
%                                \end{cases}
%\]
%である. 本論文では, この$(x - a)^n_q$に対して, 
%\[
%  (x+a)^n_q = \sum_{j=0}^n \qcoe{n}{j} q^{\frac{j(j-1)}{2}} a^j x^{n-j}
%\]
%が成り立つことの形式化を目標としている. 
あえてパラメータを増やす$q$-類似を考える利点の一つとしては, 証明が複雑な定理に対してより簡単な別証明を与えられる場合があることである. 例えば, Jacobiの三重積(\cite{Kac} p35 Theorem 11.1)
\begin{screen}
$z$, $q \in \R$, $|q| < 1$として, 
\[
  \sum_{n = -\infty}^{\infty} q^{n^2} z^n =
  \prod_{n = 1}^{\infty} (1 - q^{2n})(1 + q^{2n - 1}z)(1 + q^{2n - 1}z^{-1})
\]
が成り立つ. 
\end{screen}
はその一例である. 楕円関数論の文脈で登場する恒等式であるが(\cite{Ume} p144 (3.47)等を参照), $q$-類似で得られる式
\begin{align*}
  (1 + x)^{\infty}_q &=
    \sum_{j = 0}^{\infty} q^{j(j - 1)/2} \frac{x^j}{(1 - q) (1 - q^2) \cdots (1 - q^j)}
     \quad(\text{\cite{Kac} p30 (9.3)式}) \\
  \frac{1}{(1 - x)^{\infty}_q} &=
    \sum_{j = 0}^{\infty} \frac{x^j}{(1 - q) (1 - q^2) \cdots (1 - q^j)}
      \quad(\text{\cite{Kac} p30 (9.4)式})
\end{align*}
を用いることで簡単に証明できる. 
%%%%%%%%%%%%%%%%%%%%%%%%%%%%%%%%%%%%%%%%%%%%%%%%%%%%%%%%%%%%%%%%%%%%%%%%%%%
\section{型付き$\lambda$計算} \label{sec lambda}
%%%%%%%%%%%%%%%%%%%%%%%%%%%%%%%%%%%%%%%%%%%%%%%%%%%%%%%%%%%%%%%%%%%%%%%%%%%
\subsection{$\lambda$計算} \label{ssec lambdacal}
まず型のない$\lambda$計算を定義する. 初めに, $\lambda$計算がどのようなものなのかについての概要を説明し, その後厳密な定義に移る. \\
\subsection*{$\lambda$計算の概要}
$\lambda$計算には, 抽象と適用の2つの基本的な操作がある. まず, 抽象については, 「式から関数を作る操作」と捉えることができる. $M$を$\lambda$計算における式($\lambda$計算においてはこれを$\lambda$項と呼ぶ)だとすると, 
\[
  \lambda x . M
\]
で, 「$x$を変数とする関数」を表すことになる. 
例えば, $M$が$x^2 + 3xy + 4$という式であれば, 
\begin{itemize}
  \item $\lambda x . (x^2 + 3xy + 4)$
  \item $\lambda xy . (x^2 + 3xy + 4)$
  \item $\lambda z . (x^2 + 3xy + 4)$
\end{itemize}
はそれぞれ, 1つ目は$x \mapsto (x^2 + 3xy + 4)$という$x$についての2次関数, 
2つ目は$(x, y) \mapsto (x^2 + 3xy + 4)$という$x, y$についての2変数関数を表す. 
3つ目は, $(x^2 + 3xy + 4)$は変数$z$を含まないので, 定数関数を表すことになる. \\
もう1つの操作である適用は, 2つの$\lambda$項$M$と$N$を並べて, 
\[
  MN
\]
と書かれ, 直観的には「関数$M$に値$N$を代入する」ことを示している. 
例えば, $M$が$\lambda x . (3x + 2)$, $N$が$4$であれば, 
\[
  (\lambda x . (3x + 2))\ 4 = 3 \cdot 4 + 2\ (= 14)
\]
となる. 一般には, $[x \coloneqq N]$で$x$に$N$を代入することを表すとして, 
\[
  (\lambda x. M)\ N = M[x\coloneqq N]
\]
と書く. \\
\subsection*{$\lambda$計算の定義}
ここから実際の$\lambda$計算の定義に入る. 
\begin{shadebox}
  \begin{df}{(\cite{bib1} Definition 2.1.1)}
    $V = \{ v, v', v'', \ldots \}$を無限個の変数の集合とする. 
    $\lambda$項全体の集合$\Lambda$を, 以下のように帰納的に定義する. 
    \begin{alignat*}{3}
      &x \in V& \quad &\Lra& \quad &x \in \Lambda \\
      &M, N \in \Lambda& &\Lra& &(MN) \in \Lambda \\
      &M \in \Lambda, x \in V& &\Lra& &(\lambda x . M) \in \Lambda
    \end{alignat*}
    抽象構文を用いると, 以下のようにも書ける. 
    \begin{align*}
      &V ::= v\ |\ V' \\
      &\Lambda ::= V\ |\ (\Lambda \Lambda)\ |\ (\lambda V \Lambda)
    \end{align*}
  \end{df}
\end{shadebox}
\begin{ex}
  $v$, $(v v')$, $\lambda v (v v')$, $((\lambda v (vv''))v')$などが$\lambda$項の例である. 
\end{ex}
\begin{rmk}
表記上のルールや省略を以下のように約束する. 
\begin{enumerate}
  \item $x, y, z, \ldots$ で任意の変数を, $M, N, L, \ldots$で任意の$\lambda$項を表す. 
  \item $F M_1 \cdots M_n$で$(\cdots ((F M_1) M_2)\cdots M_n)$を, 
           $\lambda x_1 \cdots x_n .M$で
           $(\lambda x_1 (\lambda x_2 (\cdots (\lambda x_n(M))\cdots)))$
            を表す. 
  \item 最も外側の()は書かない. 
\end{enumerate}
\end{rmk}
%\begin{rmk}{(\cite{bib1} Notation 2.1.4)}
%$M, N$を$\lambda$項とする. このとき, $M$と$N$が等しいか, 束縛変数の名前をつけかえることで互いにうつりあうとき, $M \equiv N$と書く. 例えば, 
%\begin{align*}
%  (\lambda x .x) z \equiv (\lambda x .x) z \\ 
%  (\lambda x .x) z \equiv (\lambda y .y) z \\
%  (\lambda x .x) z \nequiv (\lambda x .y) z
%\end{align*}
%となる. 
%\end{rmk}
\begin{shadebox}
  \begin{df}{(\cite{bib1} Notation 2.1.4, Definition 2.1.5)}
    $M, N, P, Q \in \Lambda$, $x, y \in V$とする. 
    \begin{enumerate}
      \item $M$の自由変数の集合$FV(M)$を, 以下のように帰納的に定義する. 
        \begin{alignat*}{3}
          &FV(x)& \quad &=& \quad &\{ x \} \\
          &FV(MN)& &=& &FV(M) \cup FV(N) \\
          &FV(\lambda x. M)& &=& &FV(M) - \{ x \}
        \end{alignat*}
      \item $M$に現れる変数のうち, 自由変数でないものを束縛変数と呼ぶ. 
               また, $M$と$N$が等しいか, 束縛変数の名前をつけかえることで互いにうつりあうとき,
               $M \equiv N$と書く.
      \item $M$において$x$に$N$を代入した結果$M[x \coloneqq N]$を, 
               以下のように帰納的に定義する. 
        \begin{alignat*}{3}
        &y\ [x \coloneqq N]& \quad &\equiv& \quad &\begin{cases}
                                                                          N & x = y \\
                                                                          y & x \neq y
                                                                       \end{cases} \\
        &(PQ)\ [x \coloneqq N]& &\equiv& &(P [x \coloneqq N])\ (Q [x \coloneqq N]) \\
        &(\lambda y.M)\ [x \coloneqq N]& &\equiv& &\begin{cases}
                                                \lambda y .(M [x \coloneqq N]) & x \neq y \\
                                                \lambda y .M & x = y
                                                                       \end{cases}
        \end{alignat*}
    \end{enumerate}
  \end{df}
\end{shadebox}
\begin{rmk}{(\cite{bib1} Notation 2.1.4)}
2.について, 例えば, 
\begin{align*}
  (\lambda x .x) z \equiv (\lambda x .x) z \\ 
  (\lambda x .x) z \equiv (\lambda y .y) z \\
  (\lambda x .x) z \nequiv (\lambda x .y) z
\end{align*}
となる. 
\end{rmk}
\begin{rmk}
  束縛変数は, 自由変数とは異なる名前とする. 例えば, $y\ (\lambda y . xy)$は
  $y\ (\lambda y' . xy')$などに書き換える. 
\end{rmk}
%$\lambda$計算を, $\lambda$項同士の等式による形式体系として考えるために, 次のように定義する. 
%\begin{shadebox}
%  \begin{df}{(\cite{bib1} Definition 2.1.7)}
%    $M, M', N, L, Z \in \Lambda$とする. 
%    \begin{enumerate}
%      \item $\lambda$計算の主公理スキームを, 
%        \[
%          (\lambda x.M)\ N = M\ [x \coloneqq N]
%        \]
%        で定める. これを, $\beta$-変換という. 
%      \item $\beta$-変換の他の公理を以下のように定める.  
%        \begin{alignat*}{3}
%          &M = M& \\
%          &M = N& \quad &\Lra& \quad &N = M \\
%          &M = N, N = L& &\Lra& &M = L \\
%          &M = M'& &\Lra& &MZ = M'Z \\
%          &M = M'& &\Lra& &ZM = ZM' \\
%          &M = M'& &\Lra& &\lambda x. M = \lambda x. M' 
%        \end{alignat*}
%      \item $\lambda$計算において$M = N$が証明可能であることを, 
%        $\lambda \vdash M = N$もしくは単に$M = N$と書く. 
%    \end{enumerate}
%  \end{df}
%\end{shadebox}
最後に, $\beta$-簡約を定義する. 
\begin{shadebox}
  \begin{df}{(\cite{bib1} Definition 2.3.2)}
    $\Lambda$上の二項関係$\to_{\beta}$(1ステップ$\beta$-簡約), 
    $\thra_{\beta}$($\beta$-簡約)を, 
    %$=_{\beta}$を, 
    以下のように帰納的に定義する. 
    \begin{enumerate}
      \item \begin{enumerate}
        \item $(\lambda x .M)N \to_{\beta} M\ [x \coloneqq N]$
        \item $M \to_{\beta} N \Lra ZM \to_{\beta} ZN, MZ \to_{\beta} NZ,
                 \lambda x .M \to_{\beta} \lambda x .N$
               \end{enumerate}
      \item \begin{enumerate}
        \item $M \thra_{\beta} M$
        \item $M \to_{\beta} N \Lra M \thra_{\beta} N$
        \item $M \thra_{\beta} N, N \thra_{\beta} L \Lra M \thra_{\beta} L$
               \end{enumerate}
%      \item \begin{enumerate}
%        \item $M \thra_{\beta} N \Lra M =_{\beta} N$
%        \item $M =_{\beta} N \Lra N =_{\beta} M$
%        \item $M =_{\beta} N, N =_{\beta} L \Lra M =_{\beta} L$
%               \end{enumerate}
    \end{enumerate}
  \end{df}
\end{shadebox}
\begin{ex}
  \begin{enumerate}
    \item $(\lambda x . xyz)y \to_{\beta} yyz$である. 
    \item $(\lambda x . xy)(\lambda z . z) \to_{\beta} (\lambda z . z) y \to_{\beta} y$より, 
             $(\lambda x . xy)(\lambda z . z) \thra_{\beta} y$である. 
  \end{enumerate}
\end{ex}
%%%%%%%%%%%%%%%%%%%%%%%%%%%%%%%%%%%%%%%%%%%%%%%%%%%%%%%%%%%%%%%%%%%%%%%%%%%
\subsection{$\lama$と$\lambda$2}
%%%%%%%%%%%%%%%%%%%%%%%%%%%%%%%%%%%%%%%%%%%%%%%%%%%%%%%%%%%%%%%%%%%%%%%%%%%
\section{Coq} \label{sec coq}
%%%%%%%%%%%%%%%%%%%%%%%%%%%%%%%%%%%%%%%%%%%%%%%%%%%%%%%%%%%%%%%%%%%%%%%%%%%
\subsection{$\lambda$-cube と CIC}
%%%%%%%%%%%%%%%%%%%%%%%%%%%%%%%%%%%%%%%%%%%%%%%%%%%%%%%%%%%%%%%%%%%%%%%%%%%
\subsection{Coqの使い方} \label{ssec coq_use}
%Coq とは, 定理証明支援系の1つであり, 数学的な証明が正しいかどうか判定するプログラムである. 人間がチェックすることが難しい複雑な証明でも正しさが保証され, また証明付きプログラミングにも応用される. 例えば, 
この節ではCoqのコマンドとタクティックの使い方について述べる. ここでは, タクティックは証明の中でコンテクスト(変数や仮定)やゴール(証明すべき主張)を変形させるもの, コマンドはタクティック以外のものとして扱う. 
まず, よく使うコマンドについて説明する. 
\begin{description}
  \item[\tt Require Import]
    ライブラリを読み込むためのコマンドである.
    {\tt From mathcomp Require Import ssreflect}
    であれば, ライブラリ群 mathcomp から ssreflect を読み込んでいる.
  \item[\tt Section / End]
    {\tt Section [セクション名]}, {\tt End [セクション名]}でセクションを作ることができ, 
    そのセクション内共通のコンテキストを宣言できる. 
  \item[\tt Variable]
    {\tt Variable [変数]$\colon$[型]}で, 特定の型を持つ変数を宣言できる. 例えば\\
    {\tt Variable n$\colon$nat}\\
    で, 変数{\tt n}が自然数型{\tt nat}の要素であることを表している. 
    {\tt Section/End}コマンドと組み合わせることで, {\tt End}まで同じ意味で扱われ, 
    {\tt End}以降は効力を失う. 
    同時に複数の変数を宣言することもできる. その場合は\\
    {\tt Variables [変数] [変数] $\cdots$ [変数]$\colon$[型]}\\
    と書く(ただし, Coq にとっては{Variable}と{Variables}に違いは無い).    
  \item[\tt Hypothesis]
    {\tt Hypothesis [仮定名]$\colon$[仮定]}で仮定を置くことができる. {\tt Variable}同様, 
    {\tt Section/End}と組み合わせることで, セクション内共通の仮定を置くことができる. 
  \item[\tt Definition]
    新たに関数を定義するためのコマンドで, \\
    {\tt Definition [関数名] ([引数]$\colon$[引数の型])$\colon$[関数の型] := 
    [関数の定義式]}\\
    という形で用いる. 
    \begin{lstlisting}{Coq}
Definition dq (f : R -> R) x := f (q * x) - f x. \end{lstlisting}
    であれば, {\tt dq}が定義の名前, {\tt f}, {\tt x}が引数, {\tt R -> R}が{\tt f}の型であり, 
    {\tt f (q * x) - f x}が{\tt dq}を定義する式である. また, {\tt x}と{\tt dq}そのものの型は
    推論できるため省略できる. 
  \item[\tt Fixpoint]
    再起関数を定義するためのコマンドで, \\
    {\tt Fixpoint [関数名] ([引数]$\colon$[引数の型])$\colon$[関数の型] := 
    [定義中の関数を含む定義式]}\\
    と書く. 停止しない関数を認めてしまうと矛盾が生じるため, 停止性が保証されていない
    関数を定義することはできない. 
  \item[\tt Lemma]
    補題を宣言するためのコマンドで, \\
    {\tt Lemma [補題名] ([引数]$\colon$[引数の型])$\colon$[補題の主張]}
    という形である. 
    \begin{lstlisting}{Coq}
Lemma Dq_pow n x : x != 0 -> Dq (fun x => x ^ n) x = qnat n * x ^ (n - 1). \end{lstlisting}
    であれば, {\tt Dq\_pow}が補題名, {\tt n}, {\tt x}が引数, {\tt :}以降が補題の主張である. \\
    {\tt Lemma}の代わりに{\tt Theorem}, {\tt Corollary}等でも同じ機能をもつ. 
  \item[\tt Proof/Qed]
    {\tt Proof}は{\tt Lemma}の後に書いて補題の主張と証明を分ける(実際には省略可能で, 
    人間の見やすさのために書いている). 
    証明を完了させて{\tt Qed}を書くことで Coq に補題を登録することができ, 他の補題の
    証明に使えるようになる. 
\end{description}
次に, タクティックについて述べる. よく使われるタクティックは{\tt move}, {\tt apply}, {\tt rewrite}の3つである. 
\begin{description}
  \item[\tt move]
    {\tt move=> H}でゴールの前提に{\tt H}という名前をつけてコンテクストに移動する. 
    また{\tt move$\colon$H}で補題{\tt H}もしくはコンテクストに存在する{\tt H}をゴールの
    前提に移す. 
  \item[\tt apply]
    補題{\it lem}が{\tt P1 $\to$ P2}という形で, ゴールが{\tt P2}のとき, 
    {\tt apply {\it lem}}でゴールを{\tt P1}に変える. 
    コンテクストの仮定{\tt H}が{\tt P1 $\to P2$}であれば{\tt apply H}で同じことができる. 
  \item[\tt rewrite]
    {\it def}が定義のとき, {\tt rewrite /{\it def}}でゴールに出現している{\it def}を展開する. \\
    また, 補題{\it lem}が{\tt A = B}という形のとき, {\tt rewrite {\it lem}}でゴールに出現する
    {\tt A}を{\tt B}に書き換える(ただし, {\it lem}が{\tt H $\to$ (A = B)}という形であるとき, 
    {\tt H}がゴールに追加される). 更に, {\tt rewrite {\it lem} in H}で, コンテクストの{\tt H}に
    出現する{\tt A}を{\tt B}に書き換える. {\tt apply}と同じく, 使いたい等式が仮定にある場合も
    同じように使える. 
\end{description}
以下, 2つの具体例を用いて{\tt Proof}内でのタクティックの使い方を説明する. 
\bex{モーダスポーネンス}\\
命題$P$, $Q$について, $P \Lra Q$かつ$P$であれば, $Q$が成り立つということをCoqで証明する. 
まずこの主張を形式化すると以下の通り. 
\begin{lstlisting}{Coq}
From mathcomp Require Import ssreflect.

Lemma modus_ponens (P Q : Prop) : (P -> Q) /\ P -> Q. \end{lstlisting}
{\tt Prop}とは Coq において命題全体を表す型であり, {\tt /\textbackslash}は「かつ」を表している \\
このとき, Coqのゴールエリア(コンテクストとゴールが表示される画面)は以下の通りである. 
\begin{screen}
  \begin{lstlisting}{Coq}
    1 subgoal
    P, Q : Prop
    -------------------------------
    (P -> Q) /\ P -> Q \end{lstlisting}
\end{screen}
{\tt ---}の上がコンテクスト, 下がゴ-ルである. 1行目の{1 subgoal}はゴールが1つであることを示しており, 用いるタクティックによってはゴールが増えることもある. 
命題$P_1$, $P_2$, $P_3$について$P_1 /\backslash P_2 \to P_3$と$P_1 \to P_2 \to P_3$は同じ意味であり, この書き換えは{\tt move=> []}で行える. 実行すると以下のようにゴールエリアが変化する. 
\begin{screen}
  \begin{lstlisting}{Coq}
    1 subgoal
    P, Q : Prop
    -------------------------------
    (P -> Q) -> P -> Q \end{lstlisting}
\end{screen}
ゴールに前提{\tt (P $\to$ Q)}があるため, {\tt move=> pq}で{\tt pq}という名前をつけてコンテクストに移動する. 
\begin{screen}
  \begin{lstlisting}{Coq}
    1 subgoal
    P, Q : Prop
    pq : P -> Q
    -------------------------------
    P -> Q \end{lstlisting}
\end{screen}
まだ前提{\tt P}があるため, {\tt move=> p}で{\tt p}と名付けてコンテクストに移動する. 
\begin{screen}
  \begin{lstlisting}{Coq}
    1 subgoal
    P, Q : Prop
    pq : P -> Q
    p : P
    -------------------------------
    Q \end{lstlisting}
\end{screen}
ゴールが{\tt Q}であり, コンテクストに{\tt P $\to$ Q}という仮定{\tt pq}があるので, {\tt apply pq}でゴールを{\tt P}に書き換える. 
\begin{screen}
  \begin{lstlisting}{Coq}
    1 subgoal
    P, Q : Prop
    pq : P -> Q
    p : P
    -------------------------------
    P \end{lstlisting}
\end{screen}
ここまで来ると, ゴールが{\tt P}であり, コンテクストに{\tt P}があるため, {\tt by []}で証明を終了する. {\tt done}も{\tt by []}と同じ意味を持つ.
\begin{screen}
  \begin{lstlisting}{Coq}
  No more subgoals. \end{lstlisting}
\end{screen}
ゴールエリアに{\tt No more subgoals.}と表示されれば証明は終了であり, {\tt Qed}を書くことで補題として登録されることになる. \\
以上をまとめると次のようになる. 
\begin{lstlisting}{Coq}
Lemma modus_ponens (P Q : Prop) : (P -> Q) /\ P -> Q.
Proof.
  move=> [].
  move=> pq.
  move=> p.
  apply pq.
  by [].
Qed. \end{lstlisting}
説明のため細かく1行づつ書いたが, 複数の{\tt move}はまとめられること, あるタクティックによりゴールが自明なもの(コンテクストに存在する, {\tt A = A}である, 計算から簡単に示されるなど)になる場合はそのタクティックの前に{\tt by}をつけることで証明を終了させられることを用いれば, 次のように短くすることができる. 
\begin{lstlisting}{Coq}
Lemma modus_ponens (P Q : Prop) : (P -> Q) /\ P -> Q.
Proof.
  move=> [] pq p.
  by apply pq.
Qed. \end{lstlisting}
Coq による証明は, Curry-Howard 同型と呼ばれる, 
\begin{align*}
  \text{命題} &\leftrightarrow \text{型} \\
  \text{証明} &\leftrightarrow \text{型に要素が存在する}
\end{align*}
という対応関係に基づいている. また, 論理演算子についても, 以下のような対応がある. 
\begin{align*}
  P \text{ならば} Q &\quad P \rightarrow Q \\
  P \text{かつ} Q &\quad P \times Q \\
  P \text{または} Q &\quad  P + Q
\end{align*}
この同型をもとに上記の証明をもう一度考えてみると, $P \ra Q$と$P$という型に要素が存在することから, $Q$という型の要素を構成すればよいということである. \\
まず, 前提の要素それぞれに$pq$, $p$と名前をつける. これがプログラム中の{\tt move$\Rightarrow$ [] pq p}のことである. 
ここで, $P \to Q$という型は, 入力する値の型が$P$, 出力する値の型が$Q$であるような関数の型であるため, $P$の要素$p$に$pq$を適用することで, $Q$の要素を構成することができる. この関数適用がプログラム中の{\tt apply pq}のことである. 
\eex
\bex{代入計算}\\
自然数$m$, $n$について, 
\[
  m = 0 \Lra n + m = n
\]
という簡単な代入に関する計算を証明してみる. まず主張を形式化する.  
\begin{lstlisting}{Coq}
Lemma substitution m n : m = 0 -> n + m = n. \end{lstlisting}
このとき, ゴールエリアは以下の通りである. 
\begin{screen}
  \begin{lstlisting}{Coq}
    1 subgoal
    m, n : nat
    -------------------------------
    m = 0 -> n + m = n \end{lstlisting}
\end{screen}
まず{\tt m = 0}という前提を{\tt move=> Hm}で{\tt Hm}という名前をつけてコンテクストに移動する. 
\begin{screen}
  \begin{lstlisting}{Coq}
    1 subgoal
    m, n : nat
    Hm : m = 0
    -------------------------------
    n + m = 0 \end{lstlisting}
\end{screen}
仮定に{\tt m = 0}という等式があるため, {\tt rewrite Hm}でゴールの{\tt m}を{\tt 0}に書き換える. 
\begin{screen}
  \begin{lstlisting}{Coq}
    1 subgoal
    m, n : nat
    Hm : m = 0
    -------------------------------
    n + 0 = n \end{lstlisting}
\end{screen}
次に, {\tt n + 0}を{\tt n}に書き換えたい. mathcomp の ssrnat に{\tt addn0}という, \\
{\tt forall n : nat, n + 0 = n}\\
に対応する補題があるため, {\tt rewrite addn0}を実行する. 
\begin{screen}
  \begin{lstlisting}{Coq}
    1 subgoal
    m, n : nat
    Hm : m = 0
    -------------------------------
    n = n \end{lstlisting}
\end{screen}
このとき, ゴールは同じもの同士の等号であるため, 自明に成り立つ, つまり{\tt by []}(もしくは{\tt done})で終了する. 
\begin{screen}
  \begin{lstlisting}{Coq}
    No more subgoals.  \end{lstlisting}
\end{screen}
この証明をまとめると以下の通りである. 
\begin{lstlisting}{Coq}
Lemma substitution m n : m = 0 -> n + m = n.
Proof.
  move=> Hm.
  rewrite Hm.
  rewrite addn0.
  by [].
Qed. \end{lstlisting}
複数の{\tt rewrite}がまとめられることと{\tt by}の使い方から, より短く次のように書ける. 
\begin{lstlisting}{Coq}
Lemma substitution m n : m = 0 -> n + m = n.
Proof.
  move=> Hm.
  by rewrite Hm addn0.
Qed. \end{lstlisting}
更に, 前提が{\tt A = B}の形であるとき, {\tt move$\to$}でゴールの{\tt A}を{\tt B}に書き換えられること, {\tt ;}で異なるタクティックをつなげられることを用いれば以下のように書くこともでできる. 
\begin{lstlisting}{Coq}
Lemma substitution m n : m = 0 -> n + m = n.
Proof. by move->; by rewrite addn0. Qed. \end{lstlisting}
\eex
\brmk
  正確には, {\tt addn0}は
  \begin{lstlisting}{Coq}
right_id 0 addn \end{lstlisting}
  という補題であり, {\tt addn}は自然数同士の加法である.  
  {\tt right\_id}の定義は
  \begin{lstlisting}{Coq}
fun (S T : Type) (e : T) (op : S -> T -> S) => forall x : S, op x e = x \end{lstlisting}
  であり, 単位元を右から作用させても元のままであるということを一般的に定義している. 
  {\tt addn0}は{\tt right\_id}の{\tt e}に{\tt 0}を, {\tt op}に{\tt addn}を入れたものであるので,
  \begin{lstlisting}{Coq}
forall x : nat, addn x 0 = x \end{lstlisting}
  となる. 
\ermk
このように, Coqでの証明は, ゴールを自明な形になるまで繰り返し書き変えていくやり方が基本である. 
%%%%%%%%%%%%%%%%%%%%%%%%%%%%%%%%%%%%%%%%%%%%%%%%%%%%%%%%%%%%%%%%%%%%%%%%%%%
\subsection{mathcomp の型}
今回は mathcomp の ssrnum にある{\tt rcfType}を用いる. {\tt rcfType}とは, real closed field, つまり実閉体全体からなる型であり, この型の要素を実数として用いることにする. mathcomp はある型を構成するために他の型を用いているため, ヒエラルキー(階層構造)がある. 通常の数学において体が環の性質を, 環が群の性質を引き継ぐように, mathcomp でもより一般の型の性質を引き継いでいる. {\tt rcfType}も多くの性質を引き継いでおり, 特に環の性質, {\tt ringType}を持っていることが重要である. 実際, 形式化で利用するライブラリの補題の多くは{\tt ringType}に対するものである.

自然数型{\tt nat}
{.+1}は{\tt nat}での後続数を表す

整数型{\tt int}

%%%%%%%%%%%%%%%%%%%%%%%%%%%%%%%%%%%%%%%%%%%%%%%%%%%%%%%%%%%%%%%%%%%%%%%%%%%
\section{形式化}\label{sec form}
この定義を形式化するためまずは実数の形式化を考える. 
%%%%%%%%%%%%%%%%%%%%%%%%%%%%%%%%%%%%%%%%%%%%%%%%%%%%%%%%%%%%%%%%%%%%%%%%%%%
\subsection{$q$-微分の定義}
様々な$q$-類似を考えるにあたって, まずは微分の$q$-類似から始める. 以下, $q$を$1$でない実数とする. 
\bdf[\cite{Kac} p1 (1.1), p2 (1.5)]
  関数$f : \R \to \R$に対して, $f(x)$の$q$差分$d_q f(x)$を, 
  \[
    d_q f(x) \coloneqq f (qx) - f(x)
  \]
  と定める. 更に, $f(x)$の$q$微分を$D_q f(x)$を, 
  \[
    D_q f(x) \coloneqq \frac{d_q f(x)}{d_q x} = \frac{f(qx) - f(x)}{(q - 1) x}
  \]
  と定める. 
\edf
{\tt rcfType}を使って$q$-微分の形式化を以下のように行う. 
\begin{lstlisting}{Coq}
From mathcomp Require Import all_ssreflect all_algebra.
Import GRing.

Section q_analogue.
Local Open Scope ring_scope.
Variables (R : rcfType) (q : R).
Hypothesis Hq : q - 1 != 0.

Notation "f // g" := (fun x => f x / g x) (at level 40).

Definition dq (f : R -> R) x := f (q * x) - f x.
Definition Dq f := dq f // dq id. \end{lstlisting}
%となる. このコードの意味は大まかに以下のとおりである. 
%\begin{itemize}
%  \item 最初の2行で必要なライブラリの指定をしている. 
%  \item {\tt Variable}でそのセクション内で共通して使う変数を宣言している. 
%          {\tt R}がCoqにおける実数(正確には{\tt mathcomp}の{\tt algebra}の real closed field
%          :実閉体)の役割を果たす. ここではまだ出てきていないが, {\tt nat}が$0$を含む
%          自然数に, {\tt int}が整数に対応する. 
%  \item {\tt Hypothesis}で, $q$が1でないという仮定をしている. 使いやすさのため, 
%          $q \ne 1$ではなく$q - 1 \ne 0$という形にしている. 
%  \item {\tt Notation}で関数同士の割り算の記法を定義している. 
%  \item 2つの{\tt Definition}で$q$-差分と$q$-微分をそれぞれ定義している. 
%          {\tt $\coloneqq$}以前に定義の名前と引数, 以後に具体的な定義が書いてある. 
%          例えば$q$-差分についてであれば, {\tt d\_q}が名前, {\tt f}と{\tt x}が引数, 
%          {f (q * x) - f x}が定義である. ({\tt f}の後ろの{\tt : R $\to$ R}は{\tt f}の型である. 
%          一方, もう一つの引数である{\tt x}には型を書いていない. これは, Coqには強力な
%          型推論があるため, 推論できるものであれば型を書く必要がないためである. )
%          $D_q$の定義の中の{\tt id}は恒等関数のことである. 
%\end{itemize}
5行目の{\tt Variables (R$\colon$rcfType) (q$\colon$R).}は, {\tt R}が{\tt rcfType}の要素であること, {\tt q}が{\tt R}の要素であることを表している. {\tt rcfType}は実閉体全体を表す型であるため, {\tt R}は具体的な型ではなく, 実閉体としての性質をもつ抽象的な型として扱っている. 
\brmk
  $f$が微分可能であるとき, 
  \[
    \lim_{q\ra1} D_qf(x) = \frac{d}{dx}f(x)
  \] 
  が成り立つが, 本論文においては極限操作に関しての形式化は扱わない. 
\ermk
次に, $x ^ n$ ($n \in \Z_{\ge 0}$)を$q$-微分した際にうまく振る舞うように自然数の$q$-類似を定義する. 
\bdf[\cite{Kac} p2 (1.9)]
  $n \in \Z_{\ge 0}$に対して, $n$の$q$-類似$[n]$を, 
  \[
    [n] \coloneqq \frac{q^n - 1}{q - 1}
  \]
  と定義する. 
\edf
この$[n]$に対して, $(x^n)' = n x^{n-1}$の$q$-類似が成り立つ.
\bprop[\cite{Kac} p2 Example (1.7)]
  $n \in \Z_{>0}$について, 
  \[
    D_q x^n = [n] x ^{n - 1}
  \]
  が成り立つ. 
\eprop
\bpf
  定義に従って計算すればよく, 
  \[
    D_q x ^ n = \frac{(qx) ^ n - x ^ n}{(q - 1) x}
                 = \frac{q^n - 1}{q - 1} x ^ {n - 1}
                 = [n] x ^ {n - 1}
  \] 
\epf
この定義と補題の形式化は以下のとおりである. 
\begin{lstlisting}{Coq}
Definition qnat n : R := (q ^ n - 1) / (q - 1).

Lemma Dq_pow n x : x != 0 -> Dq (fun x => x ^ n) x = qnat n * x ^ (n - 1).
Proof.
  move=> Hx.
  rewrite /Dq /dq /qnat.
  rewrite -{4}(mul1r x) -mulrBl expfzMl -add_div; last by apply mulf_neq0.
  rewrite [in x ^ n](_ : n = (n -1) +1) //; last by rewrite subrK.
  rewrite expfzDr ?expr1z ?mulrA -?mulNr ?red_frac_r ?add_div //.
  rewrite -{2}[x ^ (n - 1)]mul1r -mulrBl mulrC mulrA.
  by rewrite [in (q - 1)^-1 * (q ^ n - 1)] mulrC.
Qed.
\end{lstlisting}
証明で用いている補題について, 例えば{\tt mul1r}は
\begin{lstlisting}{Coq}
forall R : ringType, right_id 1  *%R
\end{lstlisting}
という補題であり, 任意の{\tt ringType}を型にもつ{\tt R}について, {\tt R}上の単位元の性質を表している. ここで, 補題の{\tt R}の型は{\tt rcfType}ではなく{\tt ringType}であるが, 前述の通り{\tt rcfType}は{\tt ringType}の性質を持っているため, 今回の形式化に用いている{\tt rcfType}型の{\tt R}についてもこの補題を使うことができる. 補題名{\tt mul1r}の{\tt r}は{\tt ringType}に対する補題であることを表しており, {\tt mulrBl}, {\tt subrK}, {\tt mulrC}なども同じである. \\
また{\tt red\_frac\_r}は, 
\begin{lstlisting}{Coq}
forall x y z : R, z != 0 -> (x * z) / (y * z) = x / y \end{lstlisting}
という自分で用意した補題である. この補題の本質は$z / z = 1$という約分計算であり, mathcomp の ssralg の補題
\begin{lstlisting}{Coq}
divff : forall (F : fieldType) (x : F), x != 0 -> x / x = 1 \end{lstlisting}
を用いているため{\tt z $\neq$ 0}という仮定が必要になる. よって, {\tt Dq\_pow}にも{\tt $x \ne 0$}という前提を加えている. 今後も補題を形式化するにあたって, その証明の中で約分を行う際には$0$でないという前提を付け加えることになる. 
%\begin{itemize}
%  \item {\tt Definition}と同様, {\tt Lemma}について, {\tt $\coloneqq$}の前に補題の名前と
%          引数が, 後に補題の主張が書いてある. 今回であれば, {\tt Dq\_of\_pow}が補題の
%          名前で, {\tt n}と{\tt x}が引数である. 
%  \item {\tt Proof.}以下が補題の証明である. 
%  \item {\it def}が定義のとき, {\tt rewrite /{\it def}}で定義を展開している. 
%  \item {\it lem}が{\tt A = B}という形の補題のとき, {\tt rewrite {\it lem}}で結論に出現する
%          {\tt A}を{\tt B}に書き換えている. 他のコマンドの使い方については \cite{Hag}等を参照. 
%  \item {\tt red\_frac\_r}は, 
%           \begin{lstlisting}{Coq}
%red_frac_r : forall x y z : R, z != 0 -> x * z / (y * z) = x / y \end{lstlisting}
%           という補題である. この補題を使うため, もともとはなかった{\tt $x \ne 0$}という
%           前提を加えている. 実際, $D_q$の定義において分母に$x$が出現するので, 
%           $x$が$0$でないという前提は妥当である.  
%\end{itemize}
\brmk
  {\tt qnat}という名前であるが, 実際には{\tt n}の型は{\tt nat}ではなく{\tt int}にしている. 
  また, {\tt Dq\_of\_pow}の{\tt n}の型は{\tt int}であるため, より一般化した形での形式化に
  なっている. 
\ermk
\cite{Kac}では証明は1行で終わっているが, 形式化する場合には何倍もかかっている. これは, 積の交換法則や指数法則などの, 通常の数学では当たり前なことが自動では計算されず, {\tt rewrite mulrC}や{\tt rewrite expfzDr}というように{\tt rewrite}での書き換えを明示的に行わなければならないからである. 一般に, もとの数学の証明と比べてその形式化の方が長くなる. 
%%%%%%%%%%%%%%%%%%%%%%%%%%%%%%%%%%%%%%%%%%%%%%%%%%%%%%%%%%%%%%%%%%%%%%%%%%%
\subsection{$(x - a)^n$の$q$-類似}
続いて$(x - a)^n$の$q$-類似を定義し, その性質を調べる.  
\bdf[\cite{Kac} p8 Definition (3.4)]
  $x$, $a \in \R$, $n \in \Z_{\ge 0}$に対して, $(x - a)^n$の$q$-類似$(x - a)^n_q$を, 
  \[
  (x - a)^n_q = \begin{cases}
                      1 & \text{if}\ n = 0 \\
                      (x - a) (x - qa) \cdots (x - q^{n - 1} a) & \text{if}\ n \ge 1
                    \end{cases}
  \]
  と定義する. 
\edf
\bprop \label{Dq_qbinom_nonneg}
  $n\in\Z_{>0}$に対し, 
  \[
    D_q(x-a)^n_q = [n](x-a)^{n-1}_q
  \]
  が成り立つ. 
\eprop
\bpf
  $n$についての帰納法により示される. 
\epf
まず, $(x - a)^n_q$の定義を形式化する. 
\begin{lstlisting}{Coq}
Fixpoint qbinom_pos a n x := match n with
  | 0 => 1
  | n0.+1 => (qbinom_pos a n0 x) * (x - q ^ n0 * a)
  end. \end{lstlisting}
{\tt Fixpoint}を用いて再帰的な定義をしており, {\tt match}を使って{\tt n}が{\tt 0}かどうかで場合分けしている. 再帰で呼び出す際に引数{\tt n}が真に小さくなっているため停止性が保証されている. 補題の証明については以下の通り. 
\begin{lstlisting}{Coq}
Theorem Dq_qbinom_pos a n x : x != 0 ->
  Dq (qbinom_pos a n.+1) x =
  qnat n.+1 * qbinom_pos a n x.
Proof.
  move=> Hx.
  elim: n => [|n IH].
  - rewrite /Dq /dq /qbinom_pos /qnat.
    rewrite !mul1r mulr1 expr1z.
    rewrite opprB subrKA !divff //.
    by rewrite denom_is_nonzero.
  - rewrite (_ : Dq (qbinom_pos a n.+2) x =
                 Dq ((qbinom_pos a n.+1) **
                 (fun x => (x - q ^ (n.+1) * a))) x) //.
    rewrite Dq_prod' //.
    rewrite [Dq (+%R^~ (- (q ^ n.+1 * a))) x]/Dq /dq.
    rewrite opprB subrKA divff //; last by apply denom_is_nonzero.
    rewrite mulr1 exprSz.
    rewrite -[q * q ^ n * a]mulrA -(mulrBr q) IH.
    rewrite -[q * (x - q ^ n * a) * (qnat n.+1 * qbinom_pos a n x)]mulrA.
    rewrite [(x - q ^ n * a) * (qnat n.+1 * qbinom_pos a n x)]mulrC.
    rewrite -[qnat n.+1 * qbinom_pos a n x * (x - q ^ n * a)]mulrA.
    rewrite (_ : qbinom_pos a n x * (x - q ^ n * a) = qbinom_pos a n.+1 x) //.
    rewrite mulrA -{1}(mul1r (qbinom_pos a n.+1 x)).
    by rewrite -mulrDl -qnat_cat1.
Qed.
\end{lstlisting}
ここで{\tt elim:\,n}は{\tt n}の帰納法に対応している. \\
指数法則については, 一般には$(x - a)^{m + n} \neq (x - a)^m_q(x - a)^n_q$であり, 以下のようになる. 
\bprop[\cite{Kac} p8 (3.6)] \label{q_exp_low}
  $x,a\in\R$, $m,n\in\Z_{>0}$について, 
  \[
    (x-a)^{m+n}_q = (x-a)^m_q (x-q^ma)^n_q
  \]
  が成り立つ. 
\eprop
\bpf
  \begin{align*}
    (x-a)^{m+n}_q &= (x-a)(x-qa)\cdots(x-q^{m-1}a)
                         \times (x-q^ma)(x-q^{m+1}a)\cdots(x-q^{m+n-1})\\
                       &= (x-a)(x-qa)\cdots(x-q^{m-1}a)
                         \times (x-q^ma)(x-q(q^mx))\cdots(x-q^{n-1}(q^ma))\\
                       &= (x-a)^m_q(x-q^ma)^{n}_q
  \end{align*}
  より成立する.
\epf
この形式化は次のとおりである. 
\begin{lstlisting}{Coq}
Lemma qbinom_pos_explaw x a m n :
  qbinom_pos a (m + n) x =
    qbinom_pos a m x * qbinom_pos (q ^ m * a) n x.
Proof.
  elim: n.
  - by rewrite addn0 /= mulr1.
  - elim => [_|n _ IH].
    + by rewrite addnS /= addn0 expr0z !mul1r.
    + rewrite addnS [LHS]/= IH /= !mulrA.
      by rewrite -[q ^ n.+1 * q ^ m] expfz_n0addr // addnC.
Qed.
\end{lstlisting}
\cite{Kac}の証明では単に式変形しているが, 形式化の証明では{\tt m}, {\tt n}に関する帰納法を用いている. これは{\tt qbinom\_pos}が再帰的に定義されているため, $m$項と$n$項で分けるよりも最後の$1$項を取り出す方が簡単であり, 帰納法と相性が良いからである. 

この指数法則を用いて, $(x - a)^n_q$の$n$を負の数に拡張する. まず, \cite{Kac}の定義は
\bdf[\cite{Kac} p9 (3.7)] \label{qbinom_neg}
  $x$, $a \in \R$, $l\in\Z_{>0}$とする. このとき, 
  \[
    (x-a)^{-l}_q \coloneqq \frac{1}{(x-q^{-l}a)^l_q}
  \]
  と定める. 
\edf
であり, この形式化は, 
\begin{lstlisting}{Coq}
Definition qbinom_neg a n x := 1 / qbinom_pos (q ^ ((Negz n) + 1) * a) n x.
\end{lstlisting}
となる. ここで, {\tt Negz n}とは{\tt Negz n = - n.+1}をみたすものであるので, 
{\tt (Negz n) + 1}は{\tt -n}となり, \cite{Kac}の定義と一致する. また{\tt int}は
\begin{lstlisting}{Coq}
Variant int : Set := Posz : nat -> int | Negz : nat -> int.
\end{lstlisting}
のように定義されている. よって, {\tt int}は$0$以上か負かで場合分けできるため, 
{\tt n$\colon$int}に対して{\tt qbinom\_pos}の定義を以下のように整数に拡張する. 
\begin{lstlisting}{Coq}
Definition qbinom a n x :=
  match n with
  | Posz n0 => qbinom_pos a n0 x
  | Negz n0 => qbinom_neg a n0.+1 x
  end.
\end{lstlisting}
整数に拡張した$(x - a)^n_q$についても, 指数法則と$q$-微分はうまく振る舞う.
まず指数法則について見ていく. 
\bprop[\cite{Kac} p10 Proposition 3.2]
  $m$, $n \in \Z$について, Proposition \ref{q_exp_low}は成り立つ, つまり
  \[
    (x - a)^{m + n}_q = (x - a)^m_q (x - q^m a)^n_q
  \]
  が成り立つ. 
\eprop
\bpf
  $m$, $n$の正負で場合分けして示す. $m > 0$かつ$n > 0$の場合はすでに示しており, 
  $m = n =0$の場合は定義からすぐにわかる. その他の場合について, まず$m < 0$かつ
  $n \ge 0$の場合, $m = -m'$とおくと
  \begin{align*}
    (x - a)^m_q (x - q^m)^n_q &= (x - a)^{-m'}_q (x - q^{-m'} a)^n_q\\
                                       &= \frac{(x - q^{-m'} a)^n_q}{(x - q^{-m'} a)^{m'}_q}\\
                                       &= \begin{cases}
                                              (x - q^{m'} (q ^{-m'} a))^{n - m'}_q & n \ge m' \\
                                              \frac{1}{(x - q^n (q ^{-m'} a))^{m' - n}_q} & n < m'
                                            \end{cases}\\
												 &= (x - a)^{n - m'}_q\\
												 &= (x - a)^{n + m}_q
  \end{align*}
  というように, $n$と$m'$の大小で場合分けすることで示せる. 次に, $m \ge 0$かつ$n < 0$
  の場合, $n = - n'$として, 
  \begin{align*}
    (x - a)^m_q (x - q^m)^n_q &= (x - a)^{m}_q (x - q^m a)^{-n'}_q\\
                                       &= \begin{cases}
                                             \frac{(x - a)^{m - n'}_q (x - q^{m - n'} a)^{n'}_q}
                                                    {(x - q^{m - n'} a)^{n'}_q} & m \ge n' \\
                                             \frac{(x - a)^m_q}
                                     {(x - q^{m - n'})^{n' - m}_q (x - q^{n' - m}(q^{m - n'}a))} & m < n      
                                           \end{cases}\\
                                       &= \begin{cases}
                                              (x - a)^{m - n'} & m \ge n' \\
                                              \frac{1}{(x - q^{m - n'})^{n' - m}_q} & m < n'
                                            \end{cases}\\
                                       &=  (x - a)^{m - n'}_q = (x - a)^{m + n}_q
  \end{align*}
  となる. 最後に, $m < 0$かつ$n < 0$のとき, $m = -m'$, $n = -n'$として, 
  \begin{align*}
    (x - a)^m_q (x - q^m)^n_q &= (x - a)^{-m'}_q (x - q^{-m'})^{-n'}_q\\
                                       &= \frac{1}{(x - q^{-m'} a)^{m'}_q (x - q^{-n'-m'}a)^{n'}_q}\\
                              &= \frac{1}{(x - q^{-n'-m'}a)^{n'}_q (x - q^{n'}(q^{-m'-n'} a))^{m'}_q }\\
                                       &= \frac{1}{(x - q^{-n'-m'} a)^{n' + m'}_q}\\
                                       &= (x - a)^{-m'-n'}_q\\
                                       &= (x - a)^{m' + n'}
  \end{align*}
  となる.
\epf
この補題を形式化すると次のようになる. 
\begin{lstlisting}{Coq}
Theorem qbinom_explaw a m n x : q != 0 ->
  qbinom_denom a m x != 0 ->
  qbinom_denom (q ^ m * a) n x != 0 ->
  qbinom a (m + n) x = qbinom a m x * qbinom (q ^ m * a) n x.
Proof.
  move=> Hq0.
  case: m => m Hm.
  - case: n => n Hn.
    + by apply qbinom_pos_explaw.
    + rewrite qbinom_exp_pos_neg //.
      by rewrite addrC expfzDr // -mulrA.
  - case: n => n Hn.
    + by rewrite qbinom_exp_neg_pos.
    + by apply qbinom_exp_neg_neg.
Qed.
\end{lstlisting}
証明の構造としては, まず{\tt case$\colon$m}で{\tt m}が$0$以上か負かの場合分けを行い, 更にそれぞれの場合について{\tt case:n}で{\tt n}の場合分けを行っている. 
ここで, 前提の{\tt qbinom\_denom}の定義は
\begin{lstlisting}{Coq}
Definition qbinom_denom a n x :=
 match n with
  | Posz n0 => 1
  | Negz n0 => qbinom_pos (q ^ Negz n0 * a) n0.+1 x
  end. \end{lstlisting}
であり, 2つの前提は補題の右辺に出現する項の分母が$0$にならないということである. 
証明中に使われている補題のうち, {\tt qbinom\_exp\_pos\_neg}, {\tt qbinom\_exp\_neg\_pos}, {\tt qbinom\_exp\_neg\_neg}はそれぞれ$m \ge 0$かつ$n < 0$, $m <0$かつ$n \ge 0$, $m < 0$かつ$n < 0$のときの証明の形式化であり, 例えば{\tt qbinom\_exp\_pos\_neg}については以下の通り. 
\begin{lstlisting}{Coq}
Lemma qbinom_exp_pos_neg a (m n : nat) x : q != 0 ->
  qbinom_pos (q ^ (Posz m + Negz n) * a) n.+1 x != 0 ->
  qbinom a (Posz m + Negz n) x = qbinom a m x * qbinom (q ^ m * a) (Negz n) x.
Proof.
  move=> Hq0 Hqbinommn.
  case Hmn : (Posz m + Negz n) => [l|l]  /=.
  - rewrite /qbinom_neg mul1r.
    rewrite (_ : qbinom_pos a m x = qbinom_pos a (l + n.+1) x).
      rewrite qbinom_pos_explaw.
      have -> : q ^ (Negz n.+1 + 1) * (q ^ m * a) = q ^ l * a.
        by rewrite mulrA -expfzDr // -addn1 Negz_addK addrC Hmn.
      rewrite -{2}(mul1r (qbinom_pos (q ^ l * a) n.+1 x)) red_frac_r.
        by rewrite divr1.
      by rewrite -Hmn.
    apply Negz_transp in Hmn.
    apply (eq_int_to_nat R) in Hmn.
    by rewrite Hmn.
  - rewrite /qbinom_neg.
    have Hmn' : n.+1 = (l.+1 + m)%N.
      move /Negz_transp /esym in Hmn.
      rewrite addrC in Hmn.
      move /Negz_transp /(eq_int_to_nat R) in Hmn.
      by rewrite addnC in Hmn.
    rewrite (_ : qbinom_pos (q ^ (Negz n.+1 + 1) * (q ^ m * a)) n.+1 x 
               = qbinom_pos (q ^ (Negz n.+1 + 1) * (q ^ m * a))
                              (l.+1 + m) x).
      rewrite qbinom_pos_explaw.
      have -> : q ^ (Negz n.+1 + 1) * (q ^ m * a) =
                q ^ (Negz l.+1 + 1) * a.
        by rewrite mulrA -expfzDr // !NegzS addrC Hmn.
      have -> : q ^ l.+1 * (q ^ (Negz l.+1 + 1) * a) = a.
        by rewrite mulrA -expfzDr // NegzS NegzK expr0z mul1r.
      rewrite mulrA.
      rewrite [qbinom_pos (q ^ (Negz l.+1 + 1) * a) l.+1 x *
               qbinom_pos a m x]mulrC.
      rewrite red_frac_l //.
      have -> : a = q ^ l.+1 * (q ^ (Posz m + Negz n) * a) => //.
        by rewrite mulrA -expfzDr // Hmn NegzK expr0z mul1r.
      apply qbinom_exp_non0r.
      rewrite -Hmn' //.
    by rewrite Hmn'.
Qed.
\end{lstlisting}
この証明についての注目点としては, 
\begin{itemize}
  \item \cite{Kac}では$m$と$n'$の大小で場合分けをしていたが, 形式化では, 
           \begin{lstlisting}{Coq}
  case Hmn : (Posz m + Negz n) => [l|l]  /=. \end{lstlisting}
           として, {\tt m $-$ n'}の値を{\tt l}とおき, {\tt l}が{\tt 0}以上かどうかで
           場合分けをしている
  \item Coqでは$A = B$という等式はどの型の上でのものなのかが区別されている. 
           {\tt eq\_int\_to\_nat}という補題は{\tt int}上の等式を{\tt nat}上の等式に写している. 
\end{itemize}
などが挙げられる. \\
次に$q$-微分について見ていく.  
\bprop[\cite{Kac} p10 Proposition 3.3]
  $n \in \Z$について, 
  \[
    D_q (x - a)^n_q = [n] (x - a)^{n - 1}_q
  \]
  が成り立つ. ただし, $n$が整数の場合にも, 自然数のときと同様, $[n]$の定義は
  \[
    \frac{q^n - 1}{q - 1}
  \]
  である. 
\eprop
\bpf
  $n > 0$のときは Proposition \ref{Dq_qbinom_nonneg} であり, $n = 0$のときは$[0] = 0$からすぐにわかる. 
  $n < 0$のときは, Definition \ref{qbinom_neg}と, 商の微分公式の$q$-類似版である
  \[
    D_q \left( \frac{f(x)}{g(x)} \right) = \frac{g(x) D_q f(x) - f(x) D_q g(x)}{g(x) g(qx)} \quad
    \text{(\cite{Kac} p3 (1.13))}
  \]
  及び Proposition \ref{Dq_qbinom_nonneg}を用いて示される. 
\epf
\cite{Kac}と同じ方針で証明する. まず, $n = 0$のときは次の通り. 
\begin{lstlisting}{Coq}
Lemma Dq_qbinomn0 a x :
  Dq (qbinom a 0) x = qnat 0 * qbinom a (- 1) x.
Proof. by rewrite Dq_const qnat0 mul0r. Qed. \end{lstlisting}
ここで, {\tt Dq\_const}は
\begin{lstlisting}{Coq}
Lemma Dq_const x c : Dq (fun x => c) x = 0. \end{lstlisting}
という定数関数の$q$-微分は$0$であるという補題である. 次に, $n < 0$のときは以下のようになる. 
\begin{lstlisting}{Coq}
Theorem Dq_qbinom_neg a n x : q != 0 -> x != 0 ->
  (x - q ^ (Negz n) * a) != 0 ->
  qbinom_pos (q ^ (Negz n + 1) * a) n x != 0 ->
  Dq (qbinom_neg a n) x = qnat (Negz n + 1) * qbinom_neg a (n.+1) x.
Proof.
  move=> Hq0 Hx Hqn Hqbinom.
  destruct n.
  - by rewrite /Dq /dq /qbinom_neg /= addrK' qnat0 !mul0r.
  - rewrite Dq_quot //.
      rewrite Dq_const mulr0 mul1r sub0r.
      rewrite Dq_qbinom_pos // qbinom_qx // -mulNr.
      rewrite [qbinom_pos (q ^ (Negz n.+1 + 1) * a) n.+1 x *
                (q ^ n.+1 * qbinom_pos (q ^ (Negz n.+1 + 1 - 1) *
                  a) n.+1 x)] mulrC.
      rewrite -mulf_div.
      have -> : qbinom_pos (q ^ (Negz n.+1 + 1) * a) n x /
                    qbinom_pos (q ^ (Negz n.+1 + 1) * a) n.+1 x =
                      1 / (x - q ^ (- 1) * a).
        rewrite -(mulr1 (qbinom_pos (q ^ (Negz n.+1 + 1) * a) n x)) /=.
        rewrite red_frac_l.
          rewrite NegzE mulrA -expfzDr // addrA -addn2.
          rewrite (_ : Posz (n + 2)%N = Posz n + 2) //.
          by rewrite -{1}(add0r (Posz n)) addrKA.
        by rewrite /=; apply mulnon0 in Hqbinom.
      rewrite mulf_div.
      rewrite -[q ^ n.+1 *
                 qbinom_pos (q ^ (Negz n.+1 + 1 - 1) * a) n.+1 x *
                   (x - q ^ (-1) * a)]mulrA.
      have -> : qbinom_pos (q ^ (Negz n.+1 + 1 - 1) * a) n.+1 x *
                (x - q ^ (-1) * a) =
                qbinom_pos (q ^ (Negz (n.+1)) * a) n.+2 x => /=.
        have -> : Negz n.+1 + 1 - 1 = Negz n.+1.
          by rewrite addrK.
        have -> : q ^ n.+1 * (q ^ Negz n.+1 * a) = q ^ (-1) * a => //.
        rewrite mulrA -expfzDr // NegzE.
        have -> : Posz n.+1 - Posz n.+2 = - 1 => //.
        rewrite -addn1 -[(n + 1).+1]addn1.
        rewrite (_ : Posz (n + 1)%N = Posz n + 1) //.
        rewrite (_ : Posz (n + 1 + 1)%N = Posz n + 1 + 1) //.
        rewrite -(add0r (Posz n + 1)).
        by rewrite addrKA.
      rewrite /qbinom_neg /=.
      rewrite (_ : Negz n.+2 + 1 = Negz n.+1) // -mulf_div.
      congr (_ * _).
      rewrite NegzE mulrC /qnat -mulNr mulrA.
      congr (_ / _).
      rewrite opprB mulrBr mulr1 mulrC divff; last by rewrite expnon0.
      rewrite invr_expz (_ : - Posz n.+2 + 1 = - Posz n.+1) //.
      rewrite -addn1 (_ : Posz (n.+1 + 1)%N = Posz n.+1 + 1) //.
      by rewrite addrC [Posz n.+1 + 1]addrC -{1}(add0r 1) addrKA sub0r.
    rewrite qbinom_qx // mulf_neq0 //.
      by rewrite expnon0.
    rewrite qbinom_pos_head mulf_neq0 //.
    rewrite (_ : Negz n.+1 + 1 - 1 = Negz n.+1) //.
      by rewrite addrK.
    move: Hqbinom => /=.
    move/mulnon0.
    by rewrite addrK mulrA -{2}(expr1z q) -expfzDr.
Qed.\end{lstlisting}
非常に長くなっているが積の交換則や結合則などが多く, {\tt Dq\_quot}が商の$q$-微分公式の形式化であるため, \cite{Kac}の証明をそのまま形式化したものになっている. また, いくつかの項が$0$でないという条件がついているが, これらの項は Definition \ref{qbinom_neg} において分母に現れるため, {\tt Dq\_of\_pow}のときと同様妥当であると考えられる. これらをまとめて以下のように形式化できる. 
\begin{lstlisting}{Coq}
Theorem Dq_qbinom a n x : q != 0 -> x != 0 ->
  x - q ^ (n - 1) * a != 0 ->
  qbinom (q ^ n * a) (- n) x != 0 ->
  Dq (qbinom a n) x = qnat n * qbinom a (n - 1) x.
Proof.
  move=> Hq0 Hx Hxqa Hqbinom.
  case: n Hxqa Hqbinom => [|/=] n Hxqa Hqbinom.
  - destruct n.
    + by rewrite Dq_qbinomn0.
    + rewrite Dq_qbinom_pos //.
      rewrite (_ : Posz n.+1 - 1 = n) // -addn1.
      by rewrite (_ : Posz (n + 1)%N = Posz n + 1) ?addrK.
  - rewrite Dq_qbinom_int_to_neg Dq_qbinom_neg //.
        rewrite Negz_addK.
        rewrite (_ : (n + 1).+1 = (n + 0).+2) //.
        by rewrite addn0 addn1.
      rewrite (_ : Negz (n + 1) = Negz n - 1) //.
      by apply itransposition; rewrite Negz_addK.
    by rewrite Negz_addK addn1.
Qed.
\end{lstlisting}
{\tt case:\,n}で{\tt n}が$0$以上か負かで場合分けを, {\tt destruct n}で$0$か$1$以上かの場合分けをしており, それぞれの場合で{\tt Dq\_qbinom\_0}, {\tt Dq\_qbinom\_pos}, {\tt Dq\_qbinom\_neg}を使っていることが見て取れる. 
%%%%%%%%%%%%%%%%%%%%%%%%%%%%%%%%%%%%%%%%%%%%%%%%%%%%%%%%%%%%%%%%%%%%%%%%%%%
\subsection{関数から多項式へ} \label{ssec poly}
ここまでは\cite{Kac}の定義に沿って関数に対して$q$-微分を定義し, また関数として
$(x - a)^n_q$を定義してきたが, $q$-Taylor展開やGauss's binomial formulaを形式化するに当たって多項式に対する$q$-微分と多項式としての$(x - a)^n_q$を改めて定義する. \\
ここで, Coqにおける多項式の扱いについて説明する. 
\begin{itemize}
  \item {\tt \{poly T\}}で{\tt T}係数多項式を表す型となる. {\tt T}は{\tt ringType}で
    なくてはならないが, {\tt rcfType}は{\tt ringType}の構造を引き継いでいるため, 
    今回用いている{\tt R}に対して{\tt \{poly R\}}が定義できる. 
  \item {\tt \{poly R\}}は{\tt ringType}と{\tt lmodType}の構造を持っている. 
    よって, 前者の性質から今まで使ってきた{\tt ringType}に対する補題がそのまま使え, 後者から
    スカラー倍{\tt a *$\colon$p}(ここで{\tt a$\colon$R}, {\tt p$\colon$ \{poly R\}}である)
    も定義される. 
  \item {\tt \textbackslash poly\_(i < n) E(i)}で, 次数が$n - 1$次以下, $i$次の係数が$E(i)$
    である多項式表す. 
  \item {\tt c\%:P}で定数$c$のみからなる単項式を表す. 
  \item {\tt 'X}で変数$x$のみからなる単項式を表す. 
  \item {\tt p`\_i}で多項式$p$の$i$次の係数を表す. 
  \item {\tt size p}で多項式$p$の次数$+1$を表す. 
    {\tt size}が$0$である多項式はゼロ多項式のみであり, 
    {\tt size}が$1$である多項式は定数項のみからなる式である. 
  \item {\tt p.[x]}で多項式$p$の$x$での値を表す. 
\end{itemize}
より詳細な内容については mathcomp の poly.v を参照のこと. \\
この{\tt \{poly R\}}を用いて{\tt Dq}や{\tt qbinom}を定義し直していく. 
まず, $q$-微分について, 多項式に対する$d_q$を以下のように定義し直す. 
\begin{lstlisting}{Coq}
Definition scale_var (p : {poly R}):= \poly_(i < size p) (q ^ i * p`_i).
Definition dqp p := scale_var p - p. \end{lstlisting}
{\tt scale\_var}は多項式{\tt p}を受け取り, $i$次の係数を$q^i$倍した多項式を返す操作である. また, {\tt dqp}は多項式に対しての$d_q$と同じ結果になることが確認できる(正確には, {\tt dqp}を適用した多項式での{\tt x}での値と${\tt x} \mapsto {\tt p.[x]}$という関数に${\tt d}_{\tt q}$を適用した関数の{\tt x}での値が等しいということである). 
\begin{lstlisting}{Coq}
Definition ap_op_poly (D : (R -> R) -> (R -> R)) (p : {poly R}) := D (fun (x : R) => p.[x]).
Notation "D # p" := (ap_op_poly D p) (at level 49).
Lemma dqp_dqE p x : (dqp p).[x] = (dq # p) x. \end{lstlisting}
この{\tt dqp}を用いて, 多項式に対する$D_q$を定義する. 
\begin{lstlisting}{Coq}
Definition Dqp p := dqp p %/ dqp 'X.
\end{lstlisting}
{\tt p \%/ p'}は多項式{\tt p}を多項式{\tt p'}で割った商を表している. この定義だけでは
{\tt dqp}を{\tt dq 'X}で割った余りが$0$でない可能性があるため, $q$-微分の正しい形式化である保証がない. しかし実際に多項式に対して{\tt Dqp}を計算すると, {\tt dqp}の定義から, {\tt dqp p}は定数項が打ち消しあい, また{\tt dqp 'X}は{\tt (q - 1) * 'X}となるので割り切れるはずである. 
よってこのことを証明しておく.  
\begin{lstlisting}{Coq}
Lemma Dqp_ok p : dqp 'X %| dqp p.
\end{lstlisting}
ここで, {\tt p' \%| p}で{\tt p}が{\tt p'}で割り切れることを表す. 
今後は扱いやすさのため, {\tt `X}で約分した形
\begin{lstlisting}{Coq}
Definition Dqp' (p : {poly R}) := \poly_(i < size p) (qnat (i.+1) * p`_i.+1).
\end{lstlisting}
を用いる. このとき, {\tt Dqp}と{\tt Dqp'}が等しいことも示せる. 
\begin{lstlisting}{Coq}
Lemma Dqp_Dqp'E p : Dqp p = Dqp' p. \end{lstlisting}
また, {\tt dqp}のときと同様, 多項式に対しての$D_q$と同じであることを確認しておく. 
\begin{lstlisting}{Coq}
Lemma Dqp'_DqE p x : x != 0 -> (Dqp' p).[x] = (Dq # p) x. \end{lstlisting}
\brmk
  {\tt Dqp\_Dqp'E}には特に条件がなく, {\tt Dqp'\_DqE}には$x \neq 0$という条件がついている. 
  この違いは, 前者は{\tt 'X / 'X = 1\%:P}という多項式での約分を, 後者は{\tt x / x = 1}という
  実数での約分を行っているということから生じている. 後で詳しく述べるが, 約分の際に条件が
  必要なくなることが多項式で考える利点の一つである. 
\ermk
次に, $(x - a)^n_q$を多項式として以下のように定義しなおす. 
\begin{lstlisting}{Coq}
Fixpoint qbinom_pos_poly a n := match n with
  | 0 => 1
  | n.+1 => (qbinom_pos_poly a n) * ('X - (q ^ n * a)%:P)
  end. \end{lstlisting}
この多項式の{\tt x}での値は元の定義の{\tt qbinom\_pos}と等しくなる. 
\begin{lstlisting}{Coq}
Lemma qbinom_posE a n x :
  qbinom_pos a n x = (qbinom_pos_poly a n).[x].
\end{lstlisting}
更に, このように定義した{\tt Dqp}と{\tt qbinom\_pos\_poly}に対してもProposition \ref{Dq_qbinom_nonneg}と同じことが成り立つ. 
\begin{lstlisting}{Coq}
Lemma Dqp'_qbinom_poly a n :
  Dqp' (qbinom_pos_poly a n.+1) = (qnat n.+1) *: (qbinom_pos_poly a n).
\end{lstlisting}
\brmk
  証明の方針はこれまでの関数としての場合と同じだが, {\tt Dq\_prod'}($q$-微分の積の法則)に
  対応する補題の証明のため, {\tt scale\_var}が積について分解できること, つまり
  \begin{lstlisting}{Coq}
Lemma scale_var_prod (p p' : {poly R}) : scale_var (p * p') = scale_var p * scale_var p'. \end{lstlisting}
  を示している. ここで証明の冒頭を抜き出すと以下のようになっている. 
  \begin{lstlisting}{Coq}
Proof.
  pose n := size p.
  have : (size p <= n)%N by [].
  clearbody n.
  have Hp0 : forall (p : {poly R}), size p = 0%N ->
    scale_var (p * p') = scale_var p * scale_var p'.
    move=> p0 /eqP.
    rewrite size_poly_eq0.
    move/eqP ->.
    by rewrite mul0r scale_varC mul0r.
  elim: n p => [|n IH] p Hsize.
  ...
Qed. \end{lstlisting}
  {\tt pose n := size p.}で多項式{\tt p}の{\tt size}を{\tt n}と置いており, 
  {\tt have$\colon$(size p <= n)\%N by [].}で{\tt size p}が{\tt n}以下という自明な主張を
  あえて置いているが, これは多項式の{\tt size}に関する帰納法を用いるためである. 
  このように, 当たり前の内容を明示的に書かなければならないことに加え, 
  形式化するための証明の構造を工夫しなければならない場合もある. 
\ermk
多項式で考える理由は, $q$-Taylor展開が多項式に対する定理であることに加え, 以下の2つの目的がある. 
\begin{enumerate}
\item \textbf{$x / x = 1$の計算に$x \neq 0$という条件が必要ない}\\
  先に見たように, Coqで約分の計算, つまり$x / x = 1$を行う際には$x \ne 0$という条件が
  必要である. よって, 実数$x$対して$x / x = 1$を計算する場合, 後から$x = 0$を代入することは
  できない.   しかし, 多項式で考える場合, {\tt 'X}は単項式であるためゼロ多項式とは
  異なるので, {\tt 'X $\neq$ 0}という条件は自動的にみたされることになり, {\tt 'X / 'X = 1\%:P}
  の計算には特に条件が必要ない. よって, {\tt 'X}で約分した後でも$0$での値を計算できる.  
  例えば$D_q (x + a)^n_q = [n](x + a)^{n - 1}_q$という計算には$x$での約分が必要であるが, 
  多項式として考える場合には上の計算をした後でも$0$での値を求めることができる. 
  この値は本論文の目的であるGauss's binomial formulaの証明に必要である. 
%本論文の目的であるGauss's binomial formula は$q$-Taylor展開の系として導かれるので, $q$-Taylor展開に必要な条件はGauss's binomial formulaにも必要となる. ここで, $q$-Taylor展開の証明には$D_q (x - a)^n_q = [n] (x - a)^{n-1}_q$を用いており, 先に見たとおりこの証明の形式化には$x \neq 0$という条件が必要である($x$で約分する計算があるため). よって, Gauss's binomial formulaの証明の形式化には$x \neq 0$という条件が必要である. 一方, Gauss's binormal formulaの証明では, $f (x) = (x + a)^n_q$として, $(Dq^j f) (0)$の値を計算する必要がある. このため, 
%
%本論文の目的であるGauss's binomial formula の証明中では, $f (x) = (x + a)^n_q$として, $(Dq^j f) (0)$の値を計算する必要がある. 一方で, $q$-Taylor展開の結果も利用しなければならないが, $q$-Taylor展開の証明のためには$D_q (x - a)^n_q = [n] (x - a)^{n-1}_q$を用いており, 先に見たとおりこの証明の形式化には$x \neq 0$という条件が必要である($x$で約分する計算があるため). したがって, $D_q^j f$に$x = 0$を代入することができない. 
%
%$D_q$の定義を思い出すと, 
%\[
%  D_q = \frac{f(qx) - f(x)}{(q - 1)x}
%\]
%であった. 分母に$x$が出現するため, $x = 0$は定義域から除かれる. また, 先に見たように, 形式化の際にも計算過程で約分をする場合$x \neq 0$という条件が必要になっていた. しかし, 本論文の目的である Gauss's binomial formula の証明中において, $D_q^j f (0)$の値を計算する必要がある. 任意の点において$D_q$が定義できるように修正する必要がある. \\
\item \textbf{$q = 0$のとき高階$D_q$が定義できる}\\
	$q = 0$のときに2階$D_q$を計算してみると
	\begin{align*}
	  (D_q ^2 f) (x) = (D_0 ^2 f) (x) &= (D_0 (D_0 f)) (x) \\
	                    &= D_0 \left( \lambda x.\ \frac{f (0 x) - f(x)}{(0 - 1)x} \right) (x) \\
	                    &= D_0 \left( \lambda x.\ \frac{f(x) - f(0)}{x} \right) (x) \\
	                    &= (D_0 F) (x)
	                      \quad (\text{ここで}F \coloneqq \lambda x.\ \frac{f(x) - f(0)}{x}
	                                \text{とおいた})\\
	                    &= \lambda x.\ \frac{F(x) - F(0)}{x} (x) \\
	                    &= \frac{F(x) - F(0)}{x}
	\end{align*}
	となるが, 
	\[
	  F(0) = \frac{f(0) - f(0)}{0} = \frac{0}{0}
	\]
	となってしまう
	(Coq では$0/0$は$0$と定義されているが, これでも正しい計算結果とはならない).
	この問題が起きるのはもともとの{\tt dq}を関数の引数に対して各点ごとに定義しているから
	であり, 多項式の係数を変化させることで定義している{\tt dqp}では$q = 0$でも
	問題が起きない. よってこの{\tt dqp}を用いている{\tt Dqp}および{\tt Dqp'}については
	$q = 0$かどうかにかかわらず高階$q$-微分を定義できる. 
\end{enumerate}
%\brmk
%この{\tt Lemma}で注意すべき点は, {\tt 'X \%/ 'X = 1\%:P}という計算をする際に特に条件が必要ないことである. Coqで約分の計算を行う際には分母が$0$でないという条件が必要だが, {\tt 'X}は単項式であるため, ゼロ多項式とは異なる. よって自動的に条件がみたされることになる. 
%一方, {\tt Dqp}と{\tt Dq}が等しいことを示そうとすると, 
%\begin{lstlisting}{Coq}
%Lemma Dqp'_DqE p x : x != 0 -> (Dqp' p).[x] = (Dq # p) x.
%\end{lstlisting}
%というように$x \neq 0$という条件が必要になる. これは多項式の割り算ではなく実数の値の割り算での約分を計算する必要があるからである.
%\ermk
%%%%%%%%%%%%%%%%%%%%%%%%%%%%%%%%%%%%%%%%%%%%%%%%%%%%%%%%%%%%%%%%%%%%%%%%%%%
\subsection{$q$-Taylor展開} \label{ssec q_Taylor}
この節では, 有限次Taylor展開の$q$-類似が成り立つこと, そしてその系として Gauss's binomial formula が成り立つことを示し, 形式化する. 
まず, 一般に以下のことが成り立つことを確認しておく. 
\bthm[\cite{Kac} p5 Theorem 2.1] \label{general_Taylor}
$\K\coloneqq\R$または$\C$, $V\coloneqq\K[x]$とし, $D$を$V$上の線型作用素とする. また, 
$\{P_n(x)\}_{n=0}\subset V$ ($n=0,1,2,\cdots$)は次の三条件をみたすとする. 
  \begin{align*}
    &\textrm{(i)}\,P_0(a) = 1,\,P_n(a)=0 \quad ({}^{\forall}n\ge1)\\
    &\textrm{(ii)}\,\deg P_n = n \quad ({}^{\forall}n\ge0)\\
    &\textrm{(iii)}\,DP_n(x) = P_{n-1}(x) \quad ({}^{\forall}n\ge1), \quad D(1) = 0
  \end{align*}
ただし, $a\in\K$である. このとき, 任意の多項式$f(x)\in V$に対し, $\deg f(x)=N$とすると, 
  \[
    f(x) = \sum_{n=0}^N(D^nf)(a)P_n(x)
  \]
が成り立つ. 
\ethm
%\bpf
%  条件(ii)から$\{P_0(x), P_1(x), \ldots , P_n(x)\}$が$V$の基底であるため, 係数列$c_k$
%  が一意にとれて, 
%  \[
%    f(x) = \sum_{k = 0}^N c_k P_k(x)
%  \]
%  と表せる. このとき, 条件(i)から, $x = a$とすることで$c_0 = f(a)$が得られる. 更に, 条件(ii), (iii)と
%  $D$の線形性から, 両辺に$n$回($1 \le n \le N$)$D$を作用させることで, 
%  \[
%    (D^n f)(x) = \sum_{k = n}^N c_k D^n P_k (x) = \sum_{k = n}^N c_k P_{k - n}(x)
%  \]
%  となる. ここで$x = a$とすることで	条件(i)から
%  \[
%    c_n = (D^n f)(a) \quad (0 \le n \le N)
%  \]
%  が得られる. 
%\epf
この定理を形式化すると以下のようになる. 
\begin{lstlisting}{Coq}
Theorem general_Taylor D n P (f : {poly R}) a :
  islinear D -> isfderiv D P ->
  (P 0%N).[a] = 1 ->
  (forall n, (P n.+1).[a] = 0) ->
  (forall m, size (P m) = m.+1) ->
  size f = n.+1 ->
  f = \sum_(0 <= i < n.+1)
          ((D \^ i) f).[a] *: P i. \end{lstlisting}
記号の意味などは以下の通りである. 
\begin{itemize}
%  \item {\tt \{poly R\}}は{\tt R}係数多項式を表す型であり, {\tt p : {poly R}, a : R}に対して
%  {\tt p.[a]}で多項式{\tt p}の{\tt a}での値を, {\tt a *: p}でスカラー倍を表す. また, {\tt size p}は
%  {\tt p}の次数$+ 1$で定義されている. 
  \item {\tt islinear}, {\tt isfderiv}はそれぞれ
    \begin{lstlisting}{Coq}
Definition islinear (D : {poly R} -> {poly R}) :=
  forall a b f g, D ((a *: f) + (b *: g)) = a *: D f + b *: D g.

Definition isfderiv D (P : nat -> {poly R}) := forall n,
  match n with
  | 0 => (D (P n)) = 0
  | n.+1 => (D (P n.+1)) = P n
  end. \end{lstlisting}
    という定義であり, 前者が線形作用素であること, 後者は条件(iii)を形式化したものである. 
  \item \cite{Kac}での証明には, $\{P_0(x), P_1(x), \ldots , P_n(x)\}$が$V$の基底
    となることを用いている. これを以下のように形式化した. 
    \begin{lstlisting}{Coq}
Lemma poly_basis n (P : nat -> {poly R}) (f : {poly R}) :
  (forall m, size (P m) = m.+1) ->
  (size f <= n.+1)%N ->
  exists (c : nat -> R), f = \sum_(0 <= i < n.+1) c i *: P i. \end{lstlisting}
    この主張には係数列{\tt c}の一意性は含まれていないため, 実際には生成系であることを
    示している. 
\end{itemize}
この定理において, 
\[
  D \equiv D_q, \quad P_n \equiv \frac{(x-a)^n_q}{[n]!}
\]
(ただし, $n\in\Z_{\ge0}$に対し, $[n]!$を
\[
    [n]! \coloneqq \begin{cases}
                          1 & (n=0)\\
                          [n]\times[n-1]\times\cdots\times[1] & (n\ge1)
                        \end{cases}
\]
と定める)とすることで, 有限次Taylor展開の$q$-類似が得られる. 
\bthm[\cite{Kac} p12 Theorem 4.1] \label{q_Taylor}
$f(x)$を, $N$次の実数係数多項式とする. 任意の$c\in\R$に対し, 
  \[
    f(x) = \sum_{j=0}^N (D_q^jf)(c)\frac{(x-c)^j_q}{[j]!}
  \]
が成り立つ. 
\ethm
\bpf
$\frac{(x-a)^n_q}{[n]!}$が, $a$, $D_q$に対してTheorem \ref{general_Taylor}の三条件をみたすことを確かめればよい. (i), (ii)は$(x-a)^n_q$の定義から, (iii)はProposition\ref{Dq_qbinom_nonneg}から分かる. 
\epf
前節で準備した{\tt Dqp}, {\tt qbinom\_pos\_poly}を用いて Theorem \ref{q_Taylor}を形式化する. 
\begin{lstlisting}{Coq}
Fixpoint qfact n := match n with
  | 0 => 1
  | n.+1 => qfact n * qnat n.+1
  end.

Theorem q_Taylorp n (f : {poly R}) c :
  (forall n, qfact n != 0) ->
  size f = n.+1 ->
  f = \sum_(0 <= i < n.+1) ((Dqp' \^ i) f).[c] *: (qbinom_pos_poly c i / (qfact i)%:P).
\end{lstlisting}
{\tt Dqp}, {\tt qbinom\_pos\_poly}をもとの定義に戻したものについては, 
\begin{lstlisting}{Coq}
Theorem q_Taylor n (f : {poly R}) x c :
  q != 0 ->
  c != 0 ->
  (forall n, qfact n != 0) ->
  size f = n.+1 ->
  f.[x] =  \sum_(0 <= i < n.+1)
             ((Dq \^ i) # f) c * qbinom_pos c i x / qfact i.
\end{lstlisting}
\brmk
  約分のための$c \neq 0$という条件に加え, 高階$D_q$を扱うため前述の通り$q \neq 0$も
  必要となる. 具体的には高階{\tt Dqp'}と{\tt Dq}を一致させる補題
  \begin{lstlisting}{Coq}
Lemma hoDqp'_DqE p x n : q != 0 -> x != 0 ->
  ((Dqp' \^ n) p).[x] = ((Dq \^ n) # p) x.
Proof.
  move=> Hq0 Hx.
  rewrite /(_ # _).
  elim: n x Hx => [|n IH] x Hx //=.
  rewrite Dqp'_DqE // {2}/Dq /dq -!IH //.
  by apply mulf_neq0 => //.
Qed. \end{lstlisting}
  の証明において, {\tt IH}(Inductive Hypothesis, 帰納の仮定)を使う際に$q \ast x \neq 0$という条件が必要となる. 
\ermk
本論文の最後に, $x ^ n$と$(x - a)^n_q$にこのTaylor展開の$q$-類似を適用する. 
\blem[\cite{Kac} p12 Example (4.4)]
  $n \in \Z_{>0}$について, 
  \[
    x ^ n = \sum_{j = 0}^n \qcoe{n}{j} (x - 1)^j_q \quad
      \left( \text{ここで, }\qcoe{n}{j} \coloneqq \frac{[n]!}{[j]![n - j]!} \right)
  \]
  が成り立つ. 
\elem
\bpf
  Theorem \ref{q_Taylor}において, $f(x) = x^n$, $c = 1$とする. 
  任意の正整数$j <= n$に対して, $D_q x^n = [n] x ^{n -1}$より, 
  \[
    (D_q^j f) (x) = [n] [n - 1] \cdots [n - j +1] x^{n - j}
  \]
  となるので, 
  \[
    (D_q^j f)(1) = [n] [n - 1] \cdots [n - j +1]
  \]
  が得られる.   
\epf
\blem[\cite{Kac} p15 Example (5.5)]
  $n \in \Z_{>0}$について, 
  \[
    (x + a)^n_q = \sum_{j = 0}^n \qcoe{n}{j} q^{j (j - 1)/2} a^j x^{n - j}
  \]
  が成り立つ. この式は Gauss's binomial formula と呼ばれる. 
\elem
\bpf
  $f = (x + a)^n_q$とすると, 任意の正整数$j <= n$に対して, 
  \[
    (D_q ^j f) (x) = [n] [n - 1] [n - j + 1] (x + a)^{n - j}_q
  \]
  であり, また
  \[
    (x + a)^m_q = (x + a) (x + qa) \cdots (x + q^{m - 1} a)
  \]
  から, $(0 + a)^m_q = a \cdot qa \cdots q^{m - 1}a = q^{m (m - 1)/2}a^m$となるので, 
  \[
    (D_q^j f) (0) = [n] [n - 1] \cdots [n - j +1]q^{(n - j) (n - j - 1)/2} a^{n - j}
  \]
  が成り立つ. よって, Theorem \ref{q_Taylor} において, $f = (x + a)^n_q$, $c = 0$として, 
  \[
    (x + a)^n_q = \sum_{j = 0}^n \qcoe{n}{j} q^{(n - j) (n - j - 1)/2} a^{n - j} x^j
  \]
  が得られる. この式の右辺において$j$を$n - j$に置き換えることで, 
  \[
    \qcoe{n}{n - j} = \frac{[n]!}{[n - j]![n - (n - j)]!} = \frac{[n]!}{[j]![n - j]!} = \qcoe{n}{j}
  \]
  に注意すれば, 
  \[
    (x - a)^n_q = \sum_{j = 0}^n \qcoe{n}{j} q^{j (j - 1)/2} a^j x^{n - j}
  \]
  が成り立つ. 
\epf
この二つの等式の形式化はそれぞれ次の通り. 
\begin{lstlisting}{Coq}
Lemma q_Taylorp_pow n : (forall n, qfact n != 0) ->
  'X^n = \sum_(0 <= i < n.+1) (qbicoef n i *: qbinom_pos_poly 1 i).

Definition qbicoef n j := qfact n / (qfact j * qfact (n - j)).
Theorem Gauss_binomial a n : (forall n, qfact n != 0) ->
  qbinom_pos_poly (-a) n =
  \sum_(0 <= i < n.+1) (qbicoef n i * q ^+ (i * (i - 1))./2 * a ^+ i) *: 'X^(n - i). \end{lstlisting}
\brmk
{\tt Gauss\_binomial}は{\tt q\_Taylorp}において{\tt c = 0}として証明している. {\tt q\_Taylorp}では約分の計算をしているが, 多項式を用いて定義しているため$0$での値を計算できる. 
%\begin{lstlisting}{Coq}
%Proof.
%  move=> Hfact.
%  rewrite (q_Taylorp n (qbinom_pos_poly (-a) n) 0) //; last by rewrite qbinom_size.
%  under eq_big_nat => i /andP [_ Hi].
%    rewrite hoDqp'_qbinom0 //.
%    rewrite [(qbinom_pos_poly 0 i / (qfact i)%:P)]mulrC.
%    rewrite polyCV scalerAl scale_constpoly.
%    have -> : qbicoef n i * qfact i * q ^+ ((n - i) * (n - i - 1))./2 *
%              a ^+ (n - i) / qfact i =
%              qbicoef n i * q ^+ ((n - i) * (n - i - 1))./2 * a ^+ (n - i).
%      rewrite -!mulrA; f_equal; f_equal.
%      rewrite mulrC -mulrA; f_equal.
%      by rewrite denomK.
%    rewrite mul_polyC qbinom_x0.
%  over.
%  done.
%Qed. \end{lstlisting}
%となっており, {\tt q\_Taylor}ではなく{\tt q\_Taylorp}において{\tt c = 0}として証明している. 
%{\tt q\_Taylor}には$c \neq 0$という前提があるため, この証明には使えない. 
\ermk
%%%%%%%%%%%%%%%%%%%%%%%%%%%%%%%%%%%%%%%%%%%%%%%%%%%%%%%%%%%%%%%%%%%%%%%%%%%
\chapter{少人数クラスまとめ} \label{chap hott}
%%%%%%%%%%%%%%%%%%%%%%%%%%%%%%%%%%%%%%%%%%%%%%%%%%%%%%%%%%%%%%%%%%%%%%%%%%%
\section{はじめに}
本章では, \cite{hott}を教科書にして修士2年次に少人数クラスで学習した Homotopy Type Theory (HoTT) についてまとめる. HoTT とは, 
\begin{align*}
  \text{$a$が型$A$の要素である} &\leftrightarrow \text{$a$が空間$A$の点である} \\
  \text{$a = b$である} &\leftrightarrow \text{点$a$と点$b$の間にパスが存在する}
\end{align*}
というように, 型理論に対してホモトピー的解釈を与えたものである. \ref{sec ua}節で HoTT の大きな特徴の一つである, univalence axiom について説明する. 大雑把にいえば, univalence axiom は「型$A$と型$B$が同型ならば, $A$と$B$は等しい」という公理である. この意味を正確にとらえるため, 型同士の等しさや同型を定義していく. また, いくつかの定義や補題を準備した後, univalence axiom から関数の外延性がしたがうことを\ref{sec funext}節で確認する. 
%%%%%%%%%%%%%%%%%%%%%%%%%%%%%%%%%%%%%%%%%%%%%%%%%%%%%%%%%%%%%%%%%%%%%%%%%%%
\section{型から型を作る} \label{sec inductive}
$A$と$B$の2つの型が与えられたとき, そこから関数型$A \to B$が構成できる. このとき, 
\begin{align*}
  f : A \to B,\ a : A &\Lra f(a) : B \\
  a : A,\ b(x) : B &\Lra \lambda a. b : A \to B 
\end{align*}
である. より一般に, 型$A$と$A$上の型族$B \to \U$が与えられれば($\U$はユニバース), 依存関数型$\prod_{a : A} B(a)$が構成でき, 
\begin{align*}
  f :\prod_{a : A} B(a),\ a : A &\Lra f(a) : B(a) \\
  a : A,\ b(x) : B(x) &\Lra \lambda a. b : \prod_{a : A} B(a) 
\end{align*}
である. さらに, 既存の型から新たな型を作るやり方として, 構成規則, 導入規則, 除去規則, 計算規則の4つを与える帰納的な方法がある. 例えば, 依存和型$\sum_{x : A} B(x)$は, 
\begin{itemize}
  \item 構成規則:$A : \U$, $B : A \to \U \Lra \sum_{x : A} B(x)$
  \item 導入規則:$a : A$, $b : B(a) \Lra (a, b) : \sum_{x : A} B(x)$
  \item 除去規則:ind$_{\sum_{x : A} B(x)} : \prod_{C : (\sum_{x : A} B(x)) \to \U} 
                               \left(\prod_{a : A} \prod_{b : B(a)} C((a, b)) \right) \to 
                                      \prod_{w : \sum_{x : A} B(x)} C(w)$
  \item 計算規則:ind$_{\sum_{x : A} B(x)} (C, g, (a, b)) :\equiv g(a)(b)$
\end{itemize}
で定義できる. 除去規則は, 「任意の$w : \sum_{x : A} B(x)$について$C(w)$を示したければ, 任意の$a : A$, $b : B(a)$について$C((a, b))$を示せばよい」と読むことができる. 
ここで, Curry-Howard 同型に基づいて考えると, 「ある要素$a$とある要素$b$が等しい」という命題は, なにかしらの型と対応するはずである. よってその型 identity type を, 
\begin{itemize}
  \item 構成規則:$A : \U \Lra \_ =_A \_ : \U$
  \item 導入規則:refl$_a : \prod_{a : A} (a =_A a)$
  \item 除去規則:ind$_{=_A} : \prod_{\left( C : \prod_{(x, y : A)} (x =_A y) \to \U \right)} 
                                       \left( \prod_{(x : A)} C(x, x, {\rm refl}_x) \right) \to
                                       \prod_{(x, y : A)} \prod_{(p : x =_A y)} C(x, y, p)$
  \item 計算規則:ind$_{=_A} (C, c, x, x, {\rm refl}_x) : \equiv c(x)$
\end{itemize}
と定義する. 除去規則は, 依存和型のときと同様に考えると, 「任意の$x$, $y : A$, $x = y$について$C(x, y, p)$を示したければ, 任意の$x : A$について$C(x, x, \refl{x})$を示せばよい」となる. 
%%%%%%%%%%%%%%%%%%%%%%%%%%%%%%%%%%%%%%%%%%%%%%%%%%%%%%%%%%%%%%%%%%%%%%%%%%%
\section{型の同型}
ここで, 型と型の間の同型を定義したい. まず, 関数の間のホモトピーを定義する. 
\bdf[\cite{hott} Definition 2.4.1]
  $A : \U$, $P : A \to \U$とする. $f$, $g : \prod_{x : A} P(x)$に対して, 
  \[
    (f \sim g) :\equiv \prod_{x : A} (f(x) = g(x))
  \]
  と定める. 
\edf
次に, 「逆写像」を定義する. 
\bdf[\cite{hott} Definition 2.4.6]
  $A$, $B : \U$, $f : A \to B$とする. このとき, $f$の quasi-inverse qinv($f$)を, 
  \[
    \qinv(f) :\equiv \sum_{g : B \to A} ((f \circ g \sim \id_B) \times
                                                   (g \circ f \sim \id_A))
  \]
\edf
例えば, id$_A$の quasi-inverse はid$_A$自身である. さらに, この qinv を用いて, isequiv を, 
\begin{itemize}
  \item qinv($f$) $\to$ isequiv($f$)
  \item isequiv($f$) $\to$ qinv($f$)
  \item $e_1$, $e_2 :$isequiv($f$) ならば $e_1 = e_2$
\end{itemize}
をみたすものとして定義したい. ここでは, 
\[
  \iseq(f) :\equiv \left( \sum_{g : B \to A} (f \circ g \sim \id_B) \right) \times
                           \left( \sum_{h : B \to A} (h \circ f \sim \id_A) \right) \quad
                           \text{(\cite{hott} p73 (2.4.10))}
\]
と定めることにする. isequiv を使って型同士の同型を定義する. 
\bdf[\cite{hott} p73 (2.4.11)]
  $A$, $B : \U$について, 
  \[
    A \simeq B :\equiv \sum_{f : A \to B} \iseq(f)
  \]
  と定める. 
\edf
型の同型については, 例えば以下のような例がある. 
\bex[\cite{hott} Lemma 2.4.12]
$A$, $B$, $C : \U$について, 
\begin{itemize}
  \item $A \simeq A$
  \item $A \simeq B \to B \simeq A$
  \item $A \simeq B \to B \simeq C \to A \simeq C$
\end{itemize}
が成り立つ. 
\eex
\bex[\cite{hott} Exercise 2.10] \label{ex sigA}
  $\sum$型は「結合的」である. つまり任意の型$A$, 型族$B : A \to \U$, 
  型族上の型族$C : (\sum_{x : A} B(x)) \to \U$に対して, 
  \[
    \left( \sum_{x : A} \sum_{y : B(x)} C((x, y))\right) \simeq
    \left( \sum_{p : \sum_{x : A} B(x)} C(p) \right)
  \]
  が成り立つ. 
\eex
%%%%%%%%%%%%%%%%%%%%%%%%%%%%%%%%%%%%%%%%%%%%%%%%%%%%%%%%%%%%%%%%%%%%%%%%%%%
\section{Univalence axiom} \label{sec ua}
これまでに定義した$=$と$\simeq$を用いて, univalence axiom の主張を正しく述べる. まず, 
\[
  \textrm{idtoeqv} : \prod_{A, B : \U} (A =_{\U} B) \to (A \simeq B)
    \quad (\text{\cite{hott} p89 (2.10.2)})
\]
を定める. この関数が存在することは, path induction よりわかる(\cite{hott} Lemma 2.10.1). このidtoeqvに対して, 
\begin{shadebox}
\begin{axi}[\cite{hott} Axiom 2.10.3]
  \[
    \textrm{ua} : \prod_{A, B : \U} \iseq(\textrm{idtoeqv}(A, B))
  \]
\end{axi}
\end{shadebox}
が univalence axiom である. とくに, この公理を仮定すれば, 
\[
  (A =_{\U} B) \simeq (A \simeq B)
\]
が成り立つ. \ref{sec inductive}章のときのように, $\ua$や$\ide$を$A =_{\U} B$という型を構成する規則だと考えれば, 
\begin{itemize}
  \item 導入規則:$A : \U,\ B: \U \Lra \ua : (A \simeq B) \to (A =_{\U} B)$
  \item 除去規則:$\ide \equiv \tp^{X \mapsto X} : (A =_{\U} B) \to (A \simeq B)$
  \item (propositional)計算規則:$\ide (\ua (f), x) = f(x)$
  \item (propositional)一意性:任意の$p : A = B$について, $p = \ua (\ide (p))$
\end{itemize}
となる($\tp$の定義は\cite{hott} p72 Lemma 2.3.1). univalence axiom をみたすようなユニバース$\U$を univalent であるという.  
%%%%%%%%%%%%%%%%%%%%%%%%%%%%%%%%%%%%%%%%%%%%%%%%%%%%%%%%%%%%%%%%%%%%%%%%%%%
\section{関数の外延性} \label{sec funext}
関数$f$, $g$について, 
\[
  f = g \lra \text{任意の$f$, $g$の定義域の要素$x$に対して, $f(x) = g(x)$}
\]
という関数の外延性の公理について述べる. 正確には, まず path induction から, 
依存関数$f$, $g : \prod_{x : A} B(x)$に対して
\[
  \happ : (f = g) \to (f \sim g) \left( \prod_{x : A} (f(x) = g(x))\right)
    \quad (\text{\cite{hott} p86 (2.9.2)})
\]
という関数を定義できる. このとき, 
\baxi[\cite{hott} Axiom 2.9.3]
  任意の$A$, $B$, $f$, $g$について, $\happ$は equivalence である
\eaxi
を関数の外延性と呼ぶ. この公理から$\happ$の quasi-inverse
\[
  \textrm{funext} : \left( \prod_{x : A} (f(x) = g(x)) \right) \to (f = g)
\]
の存在が従う. この関数のことを関数の外延性と呼ぶこともある. univalence axiom のときと同じように考えると,
\begin{itemize}
  \item 導入規則:$\fune$
  \item 除去規則:$\happ$
  \item (propositional)計算規則:$\happ (\fune (h), x) = h(x)
    \quad \left(h : \prod_{x : A} (f(x) = g(x)) \right)$
  \item (propositional)一意性:$p = \fune (x \mapsto \happ(p, x)) \quad (p : f = g)$
\end{itemize}
となる. 
%%%%%%%%%%%%%%%%%%%%%%%%%%%%%%%%%%%%%%%%%%%%%%%%%%%%%%%%%%%%%%%%%%%%%%%%%%%
\section{可縮, ファイバー}
この章と次の章では, univalence axiom から関数の外延性が従うことを示すために必要な定義や補題の準備をする. まずは, 型が可縮であるということを定義する. 
\bdf[\cite{hott} Definition 3.11.1]
  $A$を型とする. 中心と呼ばれる$a : A$が存在して, 任意の$x : A$に対して$a = x$をみたす
  とき, $A$は可縮(contractible)であるという. このことを表す型$\iscont (A)$を, 
  \[
    \iscont (A) :\equiv \sum_{a : A} \prod_{x : A} (a = x)
  \]
  と定める. 
\edf
次に, ホモトピー論ではホモトピーファイバーに対応する概念を定める. 
\bdf[\cite{hott} Definition 4.2.4]
  関数$f : A \to B$の点$y : B$の上でのファイバーを
  \[
    \fib_f(y) :\equiv \sum_{x : A} (f(x) = y)
  \]
  と定める. 
\edf
このファイバーを用いて, 関数の可縮性を定義する. 
\bdf[\cite{hott} Definition4.4.1]
  関数$f : A \to B$が可縮であるとは, 任意の$y : B$に対して$\fib_f{y}$が可縮であることである. 
  このことを表す型$\iscont (f)$を, 
  \[
    \iscont (f) :\equiv \prod_{y : B} \iscont (\fib_{f}(y))
  \]
  と定める. 
\edf
このように定義した$\iscont (f)$について, 
\[
  \iscont (f) \simeq \iseq (f)
\]
が成り立つ(\cite{hott} p138 4.5節参照). よって, ある関数が equivalence であることを示すには, 可縮であることを示せば十分である. 可縮性に関するものでよく使う補題として, 以下のようなものがある. 
\blem[\cite{hott} Lemma 3.11.8] \label{lem sigeq_cont}
  任意の$A$, $a : A$について, $\sum_{x : A} (a = x)$は可縮である
\elem
\blem[\cite{hott} Lemma 3.11.9] \label{lem sig_eq}
  $P : A \to \U$を型族とする. このとき, 以下の2つが成り立つ. 
  \begin{enumerate}
    \item 各$P(x)$が可縮である, $\sum_{x : A} P(x) \simeq A$
    \item $A$が$a$を中心として可縮であるとき, $\sum_{x : A} P(x) \simeq P(a)$
  \end{enumerate}
\elem
この2つの補題と univalence axiom から示せることとして,  
\blem[\cite{hott} Lemma 4.8.1] \label{lem fib_pr_eq}
  任意の型族$B : A \to \U$について, $\pr_1 : \sum_{x : A} B(x) \to A$の$a : A$上での
  ファイバーは$B(a)$と同型である, つまり, 
  \[
    \fib_{\pr_1} (a) \simeq B(a)
  \]
  が成り立つ. 
\elem
\bpf
  \begin{align*}
    \fib_{\pr_1} (a) &\equiv \sum_{u : \sum_{x : A} B(x)} \pr_1 (u) a \\
                        &\simeq \sum_{x : A} \sum_{b : B(x)} (x = a)
                          \quad \text{(by Example \ref{ex sigA})} \\
                        &\simeq \sum_{x : A} \sum_{b : B(x)} \sum_{p : x = a} \1 \\
                        &\simeq \sum_{x : A} \sum_{p : x = a} \sum_{b : B(x)} \1 \\ 
                        &\simeq \sum_{x : A} \sum_{p : x = a} B(x) \\
                        &\simeq B(a)
                    \quad \text{(by Example \ref{ex sigA}, Lemma \ref{lem sigeq_cont}, 
                                          Lemma \ref{lem sig_eq})}
  \end{align*}
\epf
がある. この補題は, 写像はファイブレーションと同型であるというホモトピー論での基本的な結果に対応している.  
最後に, ファイバーごとの equivalence は全空間(total space)の equivalence という言葉で特徴付けられることを示す. 
%最後に、ファイバーワイズ同値が全空間の同値で特徴付けられることを示す。
%2.3 節で、型族 P : A → U は全空間 ∑(x:A) P(x) を持つ A 上の fibration と見なすことができ、その fibration は射影 pr1 : ∑(x:A) P(x) → A であることを説明する。この観点から、二つの型族 P, Q : A → U が与えられたとき、関数 f : ∏(P(x) → Q(x)) をファイバーワイズ写像またはファイバーワイズ変換と呼ぶことがある。
%このような写像は全空間上の関数を誘導する。
\bdf[\cite{hott} Definition 4.7.5]
  型族$P$, $Q : A \to \U$と依存関数$f : \prod_{x : A} P(x) \to Q(x)$
  (このような関数を fiberwise map または fiberwise transformation と呼ぶ)が与えられたとき, 
  \[
    \total (f) :\equiv \lambda w.\ (\pr_1 w, f(\pr_1 w, \pr_2 w)) :
      \sum_{x : A} P(x) \to \sum_{x : A} Q(x)
  \]
  と定める. 
\edf
\bthm[\cite{hott} Theorem 4.7.6] \label{thm_total}
  $f$が$A$上の型族$P$と$Q$の間の fiberwise transformation 
  (つまり$f : \prod_{x : A} P(x) \to Q(x)$)であり, $x : A$と$v : Q(x)$が与えられたとする. このとき, 
  \[
    \fib_{\total(f)} ((x, v)) \simeq \fib_{f(x)} (v)
  \]
  である. 
\ethm
\bpf
  \begin{align*}
    \fib_{\total(f)} ((x, v))
    &\equiv \sum_{w : \sum_{x : A} P(x)} (\pr_1 w, f(\pr_1 w, \pr_2 w)) = (x, v)\\
    &\simeq \sum_{a : A} \sum_{u : P(a)} (a, f(a, u)) = (x, v)
      \quad \text{(by Example \ref{ex sigA})}\\
    &\simeq \sum_{a : A} \sum_{u : P(a)} \sum_{p : a = x} p_* (f(a, u)) = v
      \quad \text{(by \cite{hott} Theorem 2.7.2)}\\
    &\simeq \sum_{a : A} \sum_{p : a = x} \sum_{u : P(a)} p_* (f(a, u)) = v\\
    &\simeq \sum_{g : \sum_{a : A} a = x}
                  \left( \sum_{u : P(\pr_1 (q))} (\pr_2 (q))_* (f (pr_1(q), u) = v \right)
      \quad \text{(by Example \ref{ex sigA})}\\
    &\simeq \sum_{u : P(x)} (\refl_x)_* (f(x, u) = v)
     \quad \text{(by Lemma \ref{lem sigeq_cont} and
       Lemma \ref{lem sig_eq} (ii) as center $\equiv (x, \refl_x)$)}\\
    &\simeq \sum_{u : P(x)} f (x, u) = v
  \end{align*}
\epf
fiberwise transformation $f : \prod_{x : A} P(x) \to Q(x)$が fiberwise equivalence であるということを, 任意の$x : A$について$f(x) : P(x) \to Q(x)$が equivalence であることと定めると, 以下が成り立つ. 
\bthm[\cite{hott} Theorem 4.7.7] \label{thm fibeq_totaleq}
  $f$を$A$上の型族$P$と$Q$の間の fiberwise transformation とする. このとき, 
  $f$が fiberwise equivalence であることと$\total (f)$が equivalence であることは同値である
\ethm
\bpf
  \begin{align*}
    f \text{が fiberwise equivalence} &\equiv \prod_{x : A} \iseq (f(x)) \\
                                                &\lra \prod_{x : A} \iscont (f(x)) \\
                                     &\equiv \prod_{x : A} \prod_{v : Q(x)} \iscont (\fib_{f(x)} (v)) \\
                           &\simeq \prod_{x : A} \prod_{v : Q(x)} \iscont (\fib_{\total(f)} ((x, v)))
                             \quad \text{(by Theorem \ref{thm_total})} \\
                           &\lra \prod_{w : \sum_{x : A} Q(x)} \iscont (\fib_{\total(f)} (w)) \\
                           &\equiv \iscont (\total(f)) \\
                           &\lra \iseq (\total(f))
  \end{align*}
\epf
%%%%%%%%%%%%%%%%%%%%%%%%%%%%%%%%%%%%%%%%%%%%%%%%%%%%%%%%%%%%%%%%%%%%%%%%%%%
\section{レトラクト} \label{sec retract}
この章では, 型のレトラクトと関数のレトラクトをそれぞれ定義し, 後の章で使う補題を紹介する. 
まず, 型のレトラクトと定義は以下のとおりである. 
\bdf[\cite{hott} p125]
  型$A$, $B$について, 2つの関数レトラクション$r : A \to B$とセクション$s : B \to A$が存在し, 
  更にホモトピー
  \[
    \epsilon : \prod_{y : B} (r(s(y)) = y)
  \]
  が存在するとき, $B$は$A$のレトラクトであるという. 
\edf
このレトラクトに対して, 次の補題が成り立つ. 
\blem[\cite{hott} Lemma 3.11.7] \label{lem ret_cont}
  $B$が$A$のレトラクトであり, かつ$A$が可縮であるとき, $B$も可縮である. 
\elem
次に, 関数のレトラクトを定義する. 
\bdf[\cite{hott} Definition 4.7.2] \label{def_retract}
  関数$g : A \to B$が関数$f : X \to Y$のレトラクトであるとは, 4つのホモトピー
  \begin{align*}
    &R : r \circ s \sim \id_{A} \\
    &R' : r' \circ s' \sim \id_{b} \\
    &L : f \circ s \sim s' \circ g \\
    &K : g \circ r \sim r' \circ f
  \end{align*}
  が存在するような$r : X \to A$, $s : A \to X$, $r' : Y \to B$, $s' : B \to Y$が存在し, 更に
  任意の$a : A$に対して, パス
  \[
    H(a) : K(s(a)) \cdot r'(L(a)) = g(R(a)) \cdot R'(g(a)) ^ {-1}
  \]
  が存在することをいう. 
\edf
この定義において, ホモトピー$R$, $R'$はそれぞれ$A$が$X$のレトラクト, $B$が$Y$のレトラクトであるという条件である. 
また, 型についてのレトラクションは, 上の定義において$B \equiv Y \equiv \1$としたときの特別な場合である. 関数のレトラクトについて, まず次のことが成り立つ. 
\blem[\cite{hott} Lemma 4.7.3] \label{lem_close_retract}
  関数$g : A \to B$が関数$f : X \to Y$のレトラクトであるならば, $\fib_g(b)$は$\fib_f(s'(b))$の
  レトラクトである. ただし, $s'$は Definition \ref{def_retract}のものとする. 
\elem
\bpf
  証明の概略を述べる. 
  Definition \ref{def_retract} のとおりに記号を定める. 任意の$b : B$に対して, 
  \begin{align*}
    &\varphi_b : \fib_g (b) \to \fib_f (s'(b)), \quad
      \varphi (a, p) :\equiv (s(a), L(a) \cdot s'(p)) \\
    &\psi_b : \fib_f(s'(b)) \to \fib_g (b), \quad
      \psi (x, q) :\equiv (r(x), K(x) \cdot r'(q) \cdot R'(b))
  \end{align*}
  と定義したとき, ($\sum$の induction から)任意の$(a, p) : \fib_g (b)$に対して
  $\psi_b \varphi (a, p) = (a, p)$が成り立つ, つまり
  \[
    \prod_{b : B} \prod_{a : A} \prod_{p : g(a) = b} \psi_b \varphi (a, p) = (a, p)
  \]
  を示せば良い. 詳細は\cite{hott} p140 参照のこと. 
%  初めの2つの$\prod$を入れ替え, $p$についての path induction から
%  \[
%    \prod_{a :A} \psi_{g(a)} (\varphi_{g(a)} (a, \refl_{g(a)})) = (a, \refl_{g(a)})
%  \]
%  となる. ここで, \cite{hott} Theorem 2.7.2 を用いると, $r(s(a)) = a$のパスとして$R(a)$がとれることから, 
%  \[
%    R(a)_* (K(s(a)) \cdot r'(L(a)) \cdot R'(g(a))) = \refl_{g(a)}
%  \] 
%  を示せばよいことになる. このとき, $R(a)_* \equiv \tp^{x \mapsto g(x) = g(a)} (R(a), -)$
%  であるので, Theorem 2.11.3 より, 定数関数の ap は $\refl$になることに注意すれば, 
%  \[
%    R(a)_* (K(s(a)) \cdot r'(L(a)) \cdot R'(g(a))) =
%    g(R(a))^{-1} \cdot K(s(a)) \cdot r'(L(a)) \cdot R'(g(a))
%  \]
%  となる. 
\epf
この補題から, equivalence とレトラクトに関する補題が従う. 
\bthm[\cite{hott} Theorem 4.7.4]
  $g$が equivalence な$f$のレトラクトであれば, $g$も equivalence である. 
\ethm
\bpf
  Lemma \ref{lem_close_retract} より, $g$の任意のファイバーは$f$のファイバーのレトラクトで
  あり, $\iseq (f) \lra \iscont (f)$と Lemma \ref{lem ret_cont} から$\iseq (g)$が分かる. 
\epf
%%%%%%%%%%%%%%%%%%%%%%%%%%%%%%%%%%%%%%%%%%%%%%%%%%%%%%%%%%%%%%%%%%%%%%%%%%%
\section{Univalence implies function extensionality} \label{sec funext}
最後に, univalence axiom から関数の外延性が導けることを示す.
%よって, この節では関数の外延性を仮定しない. 
証明は, まず univalence axiom から弱い関数の外延性が導けることを示し, 次に弱い関数の外延性から通常の関数の外延性が従うことを示すという2段階で行われる. 
$\U$をユニバースとし, どこで univalent を仮定しているかを明記することにする. 
\bdf[\cite{hott} Lemma 4.9.1]
  弱い関数の外延性の公理を, 任意の$A$上の型族$B : A \to \U$に対して
  \[
    \left( \prod_{x : A} \iscont(B(x)) \right) \to \iscont \left( \prod_{x :A} B(x) \right)
  \]
  が成り立つことと定義する. 
\edf
次の補題は関数の外延性を仮定すればすぐに証明できるが, 関数の外延性を仮定しなくても univalence axiom から証明できるのがポイントである. 
\blem[\cite{hott} Lemma 4.9.2] \label{lem XA_eq_XB}
  $\U$が univalent であると仮定する. 任意の$A$, $B$, $X : \U$と任意の$e : A \equiv B$
  に対して, 
  \[
    (X \to A) \simeq (X \to B)
  \]
  の同型射は, $(e(\text{の同型射}) \circ -)$で与えられる. 
\elem
\bpf
  ある$p : A = B$について$e = \idtoeqv (p)$と仮定してよい. 
  よって, path induction より $B \equiv A$, $p \equiv \refl_{A}$とすれば, 
  $e = \idtoeqv(\refl_A) = \id_A$となる. このとき, $(e \circ -)$は$(\id_A \circ -)$となり, これは
  $X \to A$上の$\id$であるので, equivalence である. 
\epf
\bcor[\cite{hott} Corollary 4.9.3] 
  $P : A \to \U$を可縮な型の族, つまり$\prod_{x : A} \iscont(P(x))$とする. このとき, 
  射影$\pr_1 : (\sum_{x : A} P (x)) \to A$は equivalence である.
  更に, $\U$が univalent であれば, 
  \[
    (\pr_1 \circ -) : \left( A \to \sum_{x : A} P(x) \right) \simeq (A \to A)
  \]
  である. 
\ecor
\bpf
  前半について, Lemma {\ref{lem fib_pr_eq}} から, $x : A$について, $\fib_{\pr_1} (x) \equiv P(x)$である. 
  $P$が可縮な型の族なので, $\iscont(\pr_1)$が成り立つ.
  後半については, Lemma \ref{lem XA_eq_XB} よりわかる. 
\epf
上記の$\alpha \equiv (\pr_1 \circ -)$のホモトピーファイバーは可縮であるため, 特に$\id_A$上でも可縮, つまり$\iscont(\fib_{\alpha} (\id_A))$である. よって, $\prod_{x : A} P(x)$が
$\fib_{\alpha} (\id_A)$のレトラクトであることを示せば, 弱い関数の外延性が univalence axiom から従うことになる. 
\bthm[\cite{hott} Theorem 4.9.4]
  $\U$が univalent であるとし, $P : A \to \U$を可縮な型の族とする. 
  $\alpha : \left( A \to \sum_{x : A} P(x) \right) \equiv (A \to A)$とすると, 
  $\prod_{x : A} P(x)$は$\fib_{\alpha} (\id_A)$のレトラクトである. つまり, $\prod_{x : A} P(x)$
  は可縮となるので, univalence axiom から弱い関数の外延性の公理が従う. 
\ethm
\bpf
  関数$\varphi : (\prod_{x : A} P(x)) \to \fib_{\alpha} (\id_A)$, 
  $\psi : \fib_{\alpha} (\id_A) \to \prod_{x : A} P(x)$をそれぞれ, 
  \begin{align*}
    \varphi (f) :\equiv (\lambda x.\ (x, f(x)), \refl_{id_A}) \\
    \psi (g, p) :\equiv \lambda x.\ \happ (p, x)_* (\pr_2(g(x)))
  \end{align*}
  と定める. このとき, 
  \begin{align*}
    \psi (\varphi (f)) &\equiv \lambda x.\ \happ (\refl_{id_A}, y)_* (\pr_2(x, f(x))) \\
                           &\equiv \lambda x.\ \refl_* (f(x)) \\
                           &\equiv \lambda x.\ f(x) \\
                           &= f
  \end{align*}
  より成り立つ. 
\epf
\bthm[\cite{hott} Theorem 4.9.5] \label{thm weakfe_to_fe}
  弱い関数の外延性から通常の関数の外延性が従う. 
\ethm
\bpf
示したいことは
  \[
    \prod_{A : \U} \prod_{P : A \to P} \prod_{f, g : \prod_{x : A} P(x)} \iseq (\happ (f, g))
  \]
  である. このとき, 
  \[
    \prod_{g : \prod_{x : A} P (x)} \iseq (\happ(f, g))
  \]
  は, $\lambda g.\ \happ (f, g)$が fiberwise equivalence であるということなので, 
  Theorem \ref{thm fibeq_totaleq} から, 
  \[
    \total (\lambda g.\ \happ (f, g)) : \sum_{g : \prod_{x : A} P (x)} (f = g) \to
                                                 \sum_{g : \prod_{x : A} P (x)} (f \sim g)
  \]
  が equivalence であればよい. ここで, Lemma \ref{lem sigeq_cont} より
  送り元の型は可縮なので, 送り先の型
  \[
    \sum_{g : \prod_{x : A} P (x)} \prod_{x : A} (f(x) = g(x))
  \]
  が可縮であれば十分である. \cite{hott} Theorem 2.15.7 の証明のうち, 関数の外延性を
  仮定しなければ, 
  \[
    \sum_{g : \prod_{x : A} P (x)} \prod_{x : A} (f(x) = g(x))
  \]
  は
  \[
    \prod_{x : A} \sum_{u : P(x)} f(x) = u
  \]
  のレトラクトであることが示せる(逆向きの合成が$\id$とホモトピックであることにのみ
  関数の外延性を使っている).さらに, $\sum_{u : P(x)} f(x) = u$は Lemma \ref{lem sigeq_cont}
  から可縮であるため, 弱い関数の外延性から$\prod_{x : A} \sum_{u : P(x)} f(x) = u$も
  可縮になる. したがって, 
  \[
    \sum_{g : \prod_{x : A} P (x)} \prod_{x : A} (f(x) = g(x))
  \]
  も可縮である.  
\epf
\brmk
  Theorem \ref{thm weakfe_to_fe} の証明には univalent axiom は用いていない. 
\ermk
%%%%%%%%%%%%%%%%%%%%%%%%%%%%%%%%%%%%%%%%%%%%%%%%%%%%%%%%%%%%%%%%%%%%%%%%%%%
\begin{thebibliography}{9}
  \bibitem{Kac} Victor Kac, Pokman Cheung, {\it{Quantum Calculus}}, Springer, 2001.
  \bibitem{coq qana} \url{https://github.com/nakamurakaoru/q-analogue}
%  \bibitem{Gar} \url{https://www.math.nagoya-u.ac.jp/~garrigue/lecture/2021_AW/                  
%                           ssrcoq2.pdf}
  \bibitem{Hag} 萩原 学/アフェルト・レナルド, {\it Coq/SSReflect/Mathcomp}, 森北出版, 
    2018
  \bibitem{coq sl} \url{https://coq.inria.fr/distrib/current/stdlib/}
  \bibitem{coq mc} \url{https://github.com/math-comp/math-comp}
%  \bibitem{coq poly} \url{https://github.com/math-comp/math-comp/blob/master/ 
%                                 mathcomp/algebra/poly.v}
  \bibitem{coq ana} \url{https://github.com/math-comp/analysis}
  \bibitem{Ume} 梅村 浩, 『楕円関数論  楕円曲線の解析学』, 東京大学出版会, 2000.
  \bibitem{Bar} H.P.Barendregt, {\it{Lambda Calculi with Types}}
  \bibitem{hott} The Univalent Foundations Program, 
                      {\it{Homotopy Type Theory: Univalent Foundations of Mathematics}}

\end{thebibliography}
\end{document}
